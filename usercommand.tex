%%% Переопределение именований, чтобы можно было и в преамбуле использовать %%%
%\renewcommand{\chaptername}{Глава}
\renewcommand{\appendixname}{Приложение} % (ГОСТ Р 7.0.11-2011, 5.7)


%%% Переопределение именований %%%
\renewcommand{\alsoname}{см. также}
\renewcommand{\seename}{см.}
\renewcommand{\headtoname}{вх.}
\renewcommand{\ccname}{исх.}
\renewcommand{\enclname}{вкл.}
\renewcommand{\pagename}{Стр.}
\renewcommand{\partname}{Часть}
\renewcommand{\abstractname}{Аннотация}
\renewcommand{\contentsname}{Оглавление} % (ГОСТ Р 7.0.11-2011, 4)
\renewcommand{\figurename}{Рисунок} % (ГОСТ Р 7.0.11-2011, 5.3.9)
\renewcommand{\tablename}{Таблица} % (ГОСТ Р 7.0.11-2011, 5.3.10)
\renewcommand{\indexname}{Предметный указатель}
\renewcommand{\listfigurename}{Список рисунков}
\renewcommand{\listtablename}{Список таблиц}
\renewcommand{\refname}{\fullbibtitle}
\renewcommand{\bibname}{\fullbibtitle}


	% Новые переменные, которые могут использоваться во всём проекте
	\newcommand{\authorbibtitle}{Публикации автора по теме диссертации}
	\newcommand{\fullbibtitle}{Список литературы} % (ГОСТ Р 7.0.11-2011, 4)
	
	% %My commands
	\newcommand{\ISP}{[\textit{ISP}\textsuperscript{+}]\xspace}
	\newcommand{\Ispp}{Isp\textsuperscript{+}\xspace}
	\newcommand{\PSI}{[\textit{PSI}\textsuperscript{+}]\xspace}
	\newcommand{\psim}{[\textit{psi}\textsuperscript{--}]\xspace}
	\newcommand{\PSIv}[1]{[\textit{PSI}\textsuperscript{+}]\up{#1}\xspace}
	\newcommand{\isp}{[\textit{isp}\textsuperscript{-}]\xspace}
	\newcommand{\Ispm}{Isp\textsuperscript{--}\xspace}
	\newcommand{\sfp}{\textit{sfp1}$\Delta$\xspace}
	\newcommand{\Saccer}{\textit{Saccharomyces cerevisiae}\xspace}
	\newcommand{\Scer}{\textit{S.~cerevisiae}\xspace}
	\newcommand{\PIN}{[\textit{PIN}\textsuperscript{+}]\xspace}
	
	\newcommand{\up}{\textsuperscript}
	\newcommand{\down}{\textsubscript}
	
	\newcommand{\mut}[1]{\textit{sup35}$^{\textit{#1}}$}
	\newcommand{\mutnm}[1]{\textit{sup35NM}$^{\textit{#1}}$}
	\newcommand{\prot}[1]{Sup35$^{#1}$}
	\newcommand{\protnm}[1]{Sup35NM$^{#1}$}
	\newcommand{\plasmid}[1]{sup35-{#1}}
	\newcommand{\REF}{(\hl{ВСТАВИТЬ ССЫЛКУ})}
	\newcommand{\comment}[1]{\hl{\# #1 \#}}
	