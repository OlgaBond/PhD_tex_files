\chapter*{Введение}							% Заголовок
\addcontentsline{toc}{chapter}{Введение}	% Добавляем его в оглавление

\subsubsection*{Актуальность темы и степень ее разработанности}
%\addcontentsline{toc}{chapter}{Актуальность темы и степень ее разработанности}
%
%К подземным грызунам относят примерно 250 видов, проводящих всю или почти всю свою жизнь под землей. Они распространены по всем континентам за исключением Австралии и Антарктиды. Уход под землю помогает избежать открытых контактов с хищниками и сильных температурных колебаний, но приводит к возникновению новых стрессовых факторов: темнота, кислородная недостаточность и гиперкапния, повышенный инфекционный фон. Адаптации к этим условиям приводят к появлению сходных морфологических признаков у филогенетически неродственных подземных грызунов: развитая сенсорная система (как компенсация отсутствия сигналов для зрительного канала восприятия), устойчивость к гипоксии и гиперкапнии, особый метаболический паттерн (сниженная температура тела, низкий уровень обмена веществ). 
%
%Подземные грызуны являются прекрасным модельным объектом эволюционной биологии для изучения адаптаций к подземному образу жизни. Более того, сравнение их с родственными наземными видами с использованием молекулярных подходов может способствовать выявлению процессов формирования адаптаций, начиная с молекулярного уровня. 
%
%Несмотря на изученность морфологических адаптаций грызунов к подземному образу жизни, молекулярные основы этого процесса остаются не до конца понятными. Особый интерес вызывает вопрос конвергентного сходства на молекулярном уровне у филогенетически независимых линий подземных грызунов. 

К подземным грызунам относят примерно 250 видов, проводящих всю или почти всю свою жизнь под землей. Они распространены по всем континентам, за исключением Австралии и Антарктиды (\cite{Fang2015}). Уход под землю помогает избежать открытых контактов с хищниками и сильных температурных колебаний, но приводит к возникновению новых стрессовых факторов: темнота, кислородная недостаточность и гиперкапния, повышенный инфекционный фон. Существование в этих условиях приводит к формированию сходных морфо-физиологических адаптаций у филогенетически далеких форм. Несмотря на их изученность, молекулярные основы адаптивных процессов остаются не до конца понятными. К настоящему моменту в открытых базах данных уже накопилось достаточное количество как последовательностей отдельных генов, так и полногеномных данных, позволяющих проводить сравнения последовательностей и поиск молекулярных следов адаптаций на различных таксономических уровнях.

Подсемейство полевочьи (Arvicolinae) -- одна из самых молодых и многочисленных групп отряда грызунов (Rodentia), распространенная практически во всех ландшафтных зонах Северного полушария. Полевки представляет собой удобную модель для изучения темпов и форм адаптивной эволюции, прежде всего связанных с роющим и подземным образом жизни. Предыдущие исследования молекулярных адаптаций к подземному образу жизни выполнялись на немногочисленных и полностью подземных представителях филогенетически далеких таксонов из разных семейств и подотрядов (слепыши, землекопы, туко-туко). Полевки же предоставляют уникальные возможности для тестирования гипотез об универсальности механизмов адаптаций за счет сравненения близкородственных пар подземных и наземных видов в пределах одного семейства. 

\subsubsection* {Цель работы}
\addcontentsline{toc}{chapter}{Цель работы}

Целью данной работы является проведение молекулярно-генетических сравнениий филогенетически независимых подземных форм подсемейства полевочьих (Arvicolinae, Cricetidae, Rodentia) и их наземных сестринских таксонов и выявление следов отбора при освоении подземной ниши на молекулярном уровне.
\vspace{0pt plus0.5fill}

Для достижения поставленной цели были сформированы следующие задачи:
\begin{enumerate}
	\item Сравнить направление и силу отбора для гена \textit{CYTB}, белок-кодирующих митохондриальных и части ядерных генов у подземных и наземных грызунов;
	\item Провести поиск параллельных аминокислотных замен в ядерных и митохондриальных генах в независимых линиях подземных полевочьих;
	\item Выявить функции генов с отличиями в силе и направлении отбора относительно наземных грызунов и параллельными аминокислотными заменами, определить биохимические процессы, в которые они вовлечены;
	\item Сравнить количество генов со следами адаптации к подземному образу жизни среди подземных представителей подсемейства Arvicolinae;
	\item Провести сравнение геномных изменений у эволюционно молодых подземных полевочьих с представителями эволюционно более древних семейств подземных грызунов. 
\end{enumerate}
%

\subsubsection*{Научная новизна}

В рамках работы впервые проведены масштабные исследования представителей подсемества полевочьи (Arvicolinae, Rodentia) для поиска следов конвергентной эволюции в филогенетически независимых линиях подземных грызунов. Исследование выполнено на нескольких уровнях: от анализа отдельного филогенетического маркера \textit{CYTB} до пула ядерных белок-кодирующих генов. Впервые проанализированы паттерны аминокислотных замен и выявлены сайты с параллельными заменами. Для всех белок-кодирующих митохондриальных и ряда ядерных (112) генов проведена оценка силы и направления отбора несколькими методами (codeml branch model, RELAX, aBSREL). В ходе выполнения работы в лаборатории эволюционной геномики и палеогеномики ЗИН РАН получено 36 новых митохондриальных геномов и более 15 транскриптомов, что представляет существенный вклад в дальнейшее изучение эволюционной истории подсемейства. 
%

\subsubsection*{Основные положения, выносимые на защиту}

\begin{enumerate}
	\item Наблюдается ослабление уровня отбора в большинстве митохондриальных белок-кодирующих генов у подземных форм полевочьих. 
	\item У подземных форм подсемейства полевочьи присутствуют параллельные аминокислотные замены в генах, которые вовлечены в процессы адаптации к низкой концентрации кислорода.
	\item Интенсивность уровня отбора на митохондриальные и ядерные гены коррелирует со степенью специализации к подземному образу жизни и не зависит от возраста таксона.
	\item Направления адаптивной изменчивости у подземных форм полевочьих имеют тот же характер, что и в других древних специализированных семейств подземных грызунов Spalacidae, Ctenomyidae, Bathyergidae.
	 
\end{enumerate}
%

\subsubsection*{Теоретическая и практическая значимости работы}
В работе получены фундаментальные данные, описывающие молекулярные адаптации к подземному образу жизни у представителей подсемейства полевочьи. Также впервые дана сравнительная характеристика различий в уровне отбора между филогенетически независимыми подземными видами. Обнаруженные гены с измененным уровнем отбора и параллельными заменами могут служить источником для более детального изучения адаптивной и эволюционной физиологии. Собранные и опубликованные в открытом доступе митохондриальные геномы и транскриптомы будут использованы в работах по филогеографии, филогении и изучении других эволюционных процессов внутри подсемейства Arvicolinae сотрудниками как Зоологического института РАН, так и учебных и научных заведений всего мира.   
Результаты исследования могут быть использованы в курсах лекций по эволюционной биологии в вузах, школах и секциях дополнительного образования.
%

\subsubsection*{Апробация результатов}

Материалы диссертации были представлены на следующих конференциях, конгрессах и мероприятиях:
\begin{itemize} 
	\item[\textbullet] Международные конференции по вычислительной биологии MCCMB (Москва, 2017, 2019);
	\item[\textbullet] XI Международная конференция по биоинформатике и системной биологии BGRS (Новосибирск, 2018); 
	\item[\textbullet] 16 Международная конференция Rodens et Spatium (Потсдам, Германия, 2018); 
	\item[\textbullet] VII Международный конгресс общества генетиков и селекционеров ВОГиС (Санкт-Петербург, 2019);
	\item[\textbullet] Семинары лабораторий териологии и эволюционной геномики и палеогеномики ЗИН РАН (2017-2021);
	\item[\textbullet] Отчетная сессия ЗИН РАН (Санкт-Петербург, 2020);
	\item[\textbullet] Семинары лаборатории эволюционной геномики факультета биоинформатики и биоинженерии МГУ (Москва, 2018, 2021);
	\item[\textbullet] 45th Federation of the European Biochemical Societies (FEBS) Congress (дистанционно, 2021);
	\item[\textbullet] XI Съезд Териологического общества при РАН (Москва, 2022);
	\item[\textbullet] Биоинформатический семинар Университета ИТМО (Санкт-Петербург, 2022).
\end{itemize}

\subsubsection*{Публикации}

По теме диссертации опубликовано 14 работ. Из них 7 статей в журналах из списка, рекомендованного ВАК, в том числе 7 на английском языке в журналах, индексируемых международными базами данных научного цитирования Scopus и Web of Science Core Collection; 7 тезисов.


\subsubsection*{Структура и объем диссертации}

Работа состоит из введения, трех глав, заключения, выводов и списка литературы. Основная часть работы изложена на 73 страницах, содержит 16 рисуноков и 13 таблиц. Список литературы включает 148 наименований, из которых 7 на русском языке и 141 -- на английском. Приложения к работе содержит 4 таблицы на 16 страницах.

\subsubsection*{Личный вклад автора}

Личный вклад автора работы состоит в сборе материала на территории Даурского заповедника (2018 г.) и Якутии (2019 г.), обработке материала, проведении всех анализов по изучению уровня и направления отбора. Подготовка публикации осуществлялась автором как самостоятельно, так и в соавторстве с коллегами. В большинстве публикаций автор является первым автором, где ему принадлежит ведущая роль как при проведении исследований, так и при подготовке рукописей (не менее 80\%).

\subsubsection*{Благодарности}

В первую очередь хочу поблагодарить мою научную руководительницу Наталью Иосифовну Абрамсон за помощь и поддержку при выполнении диссертации, ценные советы и доверие в выборе методик анализа. Отдельную благодарность хотелось бы выразить всему коллективу лаборатории эволюционной геномики и палеогеномики ЗИН РАН за неоценимый вклад в мое зоологическое образование, освоение филогенетических и филогеографических методик: Семену Бодрову, Татьяне Петровой и Евгению Генельт-Яновскому. За помощь в изучении биоинформатических подходов искренне благодарю Институт биоинформатики, а за обсуждение полученных результатов и ценные замечания -- А.В. Сморкачеву и сотрудников ИППИ РАН, ФББ МГУ и ИОГеН РАН: Надежду Потапову, Артема Касьянова, Алексея Пенина, Марию Логачеву, Егора Базыкина и А.С. Кондрашова.   

За моральную поддержку во время написания работы хочу поблагодарить в первую очередь своего супруга Станислава Александровича Бондарева, а также Ольгу Бочкареву, Александру Пантелееву, Анну Гнетневу, Анну Ганюкову, Евгения Генельт-Яновского, членов моей семьи, друзей и коллектив Института биоинформатики. Отдельную благодарность хочу выразить моей бабушке, Паненковой Галине Ильиничне, которая до конца верила, что у меня все получится. 

За помощь в организации рабочего времени выражаю огромную благодарность Татьяне Смирновой и Екатерине Копейкиной, без которых получение результатов и написание диссертации заняло бы гораздо больше времени. Отдельную благодарность выражаю композиторам компаний CD Projekt Red, Guerrilla Games и FromSoftware, под произведения которых данная работа была написана. 

Работа выполнена в рамках темы государственного задания №~AAAA–A19–119020790106–0 в лаборатории эволюционной геномики и палеогеномики ЗИН РАН. Исследования поддержаны грантами РФФИ №18-04-00730, №15-04-04602, №18-34-20118 и РНФ № 19-74-20110.

\newpage