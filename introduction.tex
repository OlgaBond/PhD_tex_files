\chapter*{Введение}							% Заголовок
\addcontentsline{toc}{chapter}{Введение}	% Добавляем его в оглавление

\subsubsection*{Актуальность темы и степень ее разработанности}
%\addcontentsline{toc}{chapter}{Актуальность темы и степень ее разработанности}
%
К подземным грызунам относят примерно 250 видов, проводящих всю или почти всю свою жизнь в подземных туннелях. Они распространены по всем континентам за исключением Австралии и Антарктиды. Уход под землю помогает избежать открытых контактов с хищниками и сильных температурных колебаний, но приводит к возникновению новых стрессовых факторов: темнота, кислородная недостаточность и гиперкапния, нехватка пищи, повышенный инфекционный фон. Адаптации к этим условиям приводят к появлению сходных морфологических признаков у филогенетически неродственных подземных грызунов: развитая сенсорная система (как компенсация отсутствия сигналов для зрительного канала восприятия), устойчивость к гипоксии и гиперкапнии, особый метаболический паттерн (сниженная температура тела, низкий уровень обмена веществ). 

Подземные грызуны являются прекрасным модельным объектом эволюционной биологии для изучения адаптаций к нестандартным условиям обитания. Более того, сравнение их с родственными наземными видами с использованием молекулярных подходов может способствовать выявлению процессов формирования адаптаций, начиная с молекулярного уровня. 

Несмотря на изученность морфологических адаптаций грызунов к подземному образу жизни, молекулярные основы этого процесса остаются не до конца понятными. Особый интерес вызывает вопрос их схожести на молекулярном уровне у филогенетически независимых линий подземных грызунов. 

\subsubsection* {Цель работы}
\addcontentsline{toc}{chapter}{Цель работы}

Целью данной работы является проведение молекулярно-генетических сравнениий филогенетически независимых подземных форм подсемейства полевочьих (Arvicolinae, Cricetidae, Rodentia) и их наземных сестринских таксонов и выявление следов отбора при освоении подземной ниши на молекулярном уровне.
\vspace{0pt plus0.5fill}

Для достижения поставленной цели были сформированы следующие задачи:
\begin{enumerate}
	\item Оценить уровень отбора для гена \textit{CYTB}, митохондриальных и ядерных генов и сравнить его у подземных и наземных грызунов;
	\item Изучить аминокислотные замены в ядерных и митохондриальных генах и провести поиск общих паттернов;
	\item Выявить функции обнаруженных генов и биохимические процессы, в которые они вовлечены;
	\item Сравнить уровень обнаруженных изменений среди подземных представителей подсемейства Arvicolinae;
	\item Провести сравнение геномных изменений у эволюционно молодых подземных полевочьих с эволюционно более древними подземными семействами подземных грызунов. 
\end{enumerate}
%

\subsubsection*{Научная новизна}

В рамках работы впервые проведены масштабные исследования подсемества половочьи (Arvicolinae, Rodentia) для поиска следов конвергентной эволюции у филогенетически независимых подземных линий грызунов. Работа сделана на нескольких уровнях данных: от анализа отдельного филогенетического маркера \textit{CYTB} на до пула ядерных белок-кодирующих генов. Впервые проанализированы амикислотные паттерные замен и выявлены сайты с конвергентными аминокислотными заменами. Для всех генов была проведена оценка уровня отбора несколькими методами (codeml branch model, RELAX, aBSREL). В ходе выполнения работы в группе молекулярной систематики ЗИН РАН было получено 36 новых митохондриальных геномов и более 15 транскриптомов, что составляет рекордное для подсемейства количество такого типа данных. 
%

\subsubsection*{Основные положения, выносимые на защиту}

\begin{enumerate}
	\item Частота несинонимичных замен в большинстве митохондриальных белок-кодирующих генах достоверно выше у подземных форм полевочьих по сравнению с наземными. 
	\item У подземных полевочьих есть параллельные аминокислотные замены в генах, которые вовлечены в адаптации к низкой концентрации кислорода.
	\item Интенсивность отбора на митохондриальные и ядерные гены коррелирует с со степенью специализации к подземному образу жизни и не зависит от возраста таксона.
	\item Направления адаптивной изменчивости у подземных форм полевочьих имеют тот же характер, что и в других древних специализированных подземных грызунов семейств Spalacidae, Ctenomyidae, Bathergidae.
	 
\end{enumerate}
%

\subsubsection*{Теоретическая и практическая значимости работы}
В работе получены фундаментальные данные, описывающие молекулярные адпатции к подземному образу жизни у представителей подсемейства полевочьи. Также впервые дана сравнительная характеристика различий в уровне отбора между филогенетически независимыми подземными видами. Обнаруженные гены с измененным уровнем отбора и параллельными заменами могут служить источником для более детального изучения адаптивной и эволюционной физиологии. Собранные и опубликованные в открытом доступе митохондриальные геномы и транскриптомы будут использованы в работах по филогеографии, филогении и изучении других эволюционных процессов внутри подсемейства Arvicolinae сотрудниками как Зоологического института РАН, так и учебных и научных заведений всего мира.   
Результаты исследования могут быть использованы в курсах лекций по эволюционной биологии в вузах, школах и секциях дополнительного образования.
%

\subsubsection*{Апробация результатов}

Материалы диссертации были представлены на следующих конференциях, конгрессах и мероприятиях:
\begin{itemize} 
	\item[\textbullet] Международные конференциии по вычислительной биологии MCCMB (Москва, 2017, 2019);
	\item[\textbullet] XI Международная конференция по биоинформатике и системной биологии BGRS (Новосибирск, 2018); 
	\item[\textbullet] 16 Международная конференция Rodens et Spatium (Потсдам, Германия, 2018); 
	\item[\textbullet] VII Международный конгресс общества генетиков и селекционеров ВОГиС (Санкт-Петербург, 2019);
	\item[\textbullet] Семинары лаборатории териологии и группы молекулярной систематики ЗИН РАН (2017-2021);
	\item[\textbullet] Отчетная сессия ЗИН РАН (Санкт-Петербург, 2020);
	\item[\textbullet] Семинары лаборатории эволюционной геномики факультета биоинформатики и биоинженерии МГУ (Москва, 2018, 2021);
	\item[\textbullet] 45th Federation of the European Biochemical Societies (FEBS) Congress (дистанционно, 2021).
\end{itemize}

\subsubsection*{Публикации}
Основные положения диссертации изложены в пяти опубликованных и двух готовящихся к публикации печатных работах в журналах, индексируемых Web of Science и Scopus.
\begin{itemize} 
	\item[\textbullet] Bondareva O. V., Abramson N. I. The complete mitochondrial genome of the common pine vole \textit{Terricola subterraneus} (Arvicolinae, Rodentia) //Mitochondrial DNA Part B. – 2019. – Т. 4. – №. 2. – С. 3925-3926;
	\item[\textbullet] Abramson N. I. et al. Phylogenetic relationships and taxonomic position of genus \textit{Hyperacrius} (Rodentia: Arvicolinae) from Kashmir based on evidences from analysis of mitochondrial genome and study of skull morphology //PeerJ. – 2020. – Т. 8. – С. e10364;
	\item[\textbullet] Bondareva O. V. et al. The complete mitochondrial genomes of three \textit{Ellobius} mole vole species (Rodentia: Arvicolinae) //Mitochondrial DNA Part B. – 2020. – Т. 5. – №. 3. – С. 2485-2487. 
	\item[\textbullet] Bondareva O. V. et al. Searching for signatures of positive selection in cytochrome b gene associated with subterranean lifestyle in fast-evolving arvicolines (Arvicolinae, Cricetidae, Rodentia) //BMC Ecology and Evolution. – 2021. – Т. 21. – №. 1. – С. 1-12.
	\item[\textbullet] O. Bondareva, S. Bodrov, E. Genelt-Yanovskiy, T. Petrova, N. Abramson. Signatures of selection and adaptation to subterranean lifestyle across thetranscriptomes of Arvicolinae (Rodentia, Cricetidae)// FEBS Open Bio. - 2021. - 11:P-01.3-17. doi:10.1002/2211-5463.13205
	\item[\textbullet] Abramson N. I. et al. Mitochondrial genome phylogeny of voles and lemmings (Rodentia: Arvicolinae): evolutionary and taxonomic implications //bioRxiv. – 2021 (in press)
	\item[\textbullet]  Olga Bondareva, Evgeny Genelt-Yanovskiy, Tatyana Petrova, Semen Bodrov, Antonina Smorkatcheva, Natalia Abramson. Signatures of adaptation in mitochondrial genomes of the Palearctic subterranean voles (Arvicolinae, Rodentia) // MDPI Genes (in press).
\end{itemize}


\subsubsection*{Структура и объем диссертации}

Работа состоит из введения, трех глав, заключения, выводов и списка литературы. Основная часть работы изложена на 89 страницах, содержит 14 рисуноков и 10 таблиц. Список литературы включает 146 наименований, из которых 2 на русском языке и 144 -- на английском. Приложения к работе содержит 4 таблицы на 9 страницах.

\subsubsection*{Благодарности}

В первую очередь хочу поблагодарить мою научную руководительницу Наталью Иосифовну Абрамсон за помощь и поддержку при выполнении диссертации, ценные советы и доверие в выборе методик анализа. Отдельную благодарность хотелось бы выразить всему коллективу лаборатории эволюционной геномики и палеогеномики ЗИН РАН за неоценимый вклад в мое зоологическое образование, освоение филогенетических и филогеографических методик: Семену Бодрову, Татьяне Петровой и Евгению Генельт-Яновскому. За помощь в изучении биоинформатических подходов искренне благодарю Институт биоинформатики, а за обсуждение полученных результатов и ценные замечания -- А.В. Сморкачеву и сотрудников ИППИ РАН, ФББ МГУ и ИОГеН РАН: Надежду Потапову, Артема Касьянова, Алексея Пенина, Марию Логачеву, Егора Базыкина и А.С. Кондрашова.   

За моральную поддержку во время написания работы хочу поблагодарить в первую очередь своего супруга Станислава Александровича Бондарева, а также Евгения Генельт-Яновского, Ольгу Бочкареву, Александру Пантелееву, Анну Гнетневу, Анну Ганюкову, членов моей семьи, друзей и коллектив Института биоинформатики.

За помощь в организации рабочего времени выражаю огромную благодарность Татьяне Смирновой и Екатерине Копейкиной, без которых получение результатов и написание диссертации заняло бы гораздо больше времени. Отдельную благодарность выражаю композиторам компаний CR Project Red, Guerrilla Games и FromSoftware, под произведения которых данная работа была написана. 

Работа выполнена в рамках темы государственного задания № AAAA–A19–119020790106–0 в группе молекулярной систематики лаборатории териологии ЗИН РАН. Исследования поддержаны грантами РФФИ №18-04-00730, №15-04-04602, №18-34-20118 и РНФ № 19-74-20110.

\newpage