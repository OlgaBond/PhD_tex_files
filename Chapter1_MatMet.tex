\chapter{Материалы и методы} \label{mm}

\section{Материал}

Работа включает в себя поиск следов отбора к подземному образу жизни на нескольких уровнях: отдельном митохондриальном филогенетическом маркере \textit{CYTB}, полных митохондриальных геномах и ядерных генах. Количество видов, взятых в анализ на каждом из этапов, различается из-за доступности материала и количества данных, выложенных в открытый доступ в базе данных GenBank. Ген \textit{CYTB} был выбран как самый распространенный филогенетический маркер и, следовательно, дал возможность взять в анализ большое количество подземных и наземных видов.
В каждом случае мы использовали максимально репрезентативную филогенетическую выборку из доступных на момент исследования видов для адекватного таксономического контекста. Все виды, использованные в каждом из анализов, ссылки на номера сиквенсов в базе данных GenBank и других базах приведены в таблице \ref{Matheril_Table}. 

\section{Выделение ДНК}

Образцы мышечной и кожной ткани хранили в 96\% этаноле при -20 $^\circ$С в коллекции тканей и ДНК лаборатории эволюционной геномики и палеогеномики Зоологического института РАН. Все сиквенсы, проанализированные в рамках данной работы, были получены из материала коллекции или из открытой базы данных GenBank (https://www.ncbi.nlm.nih.gov/genbank/).

\subsection{Для амплификации отдельных генов методом Сенгера}

Геномную ДНК выделяли с использованием стандартного протокола солевой экстракции (\cite{Miller1999}). Выделение ДНК проводили в специально оборудованной лаборатории эволюционной геномики и палеогеномики ЗИН РАН на Английском проспекте, 32 (Санкт-Петербург, Россия).

\subsection{Для секвенирования методом NGS}

Гомогенизацию тканей выполняли с использованием ступки и пестика или Qiagen TissueLyser LT (Quiagen). Геномную ДНК экстрагировали с помощью Diatom DNA Prep 200 (Isogen, Россия). Выделение ДНК проводилось в помещении, изолированном от пост-ПЦР-установок, с использованием рабочей станции для ПЦР (LAMSYSTEMS CC) и рабочей поверхности, все инструменты и пластмассы были стерилизованы УФ-светом и хлорамином-Т, чтобы избежать загрязнения. Ультразвуковая фрагментация общей геномной ДНК проводилась с использованием сфокусированного ультразвукового прибора Covaris S220 (Covaris). Полученную фрагментированную ДНК очищали и концентрировали с использованием парамагнитной химии на основе гранул AMPure XP (Beckman-Coulter), применяя стандартные протоколы. Концентрацию ДНК оценивали с помощью флуориметра Qubit (Thermo Fisher). Выделение ДНК проводили в центре коллективного пользования в области геномики Сколтеха (https://www.skoltech.ru/research/en/shared-resources/gcf-2/). 

\section{Выделение РНК}

Образцы смешанных тканей (мышцы, печень, селезенка, семенники) от пойманных в ловушки животных хранили в фиксаторе intactRNA (Евроген, Россия) в коллекции тканей и ДНК лаборатории эволюционной геномики и палеогеномики  ЗИН РАН. Все сиквенсы, проанализированные в рамках данной работы, были получены из материала коллекции лаборатории эволюционной геномики и палеогеномики или открытой базы данных GenBank (https://www.ncbi.nlm.nih.gov/genbank/).

Выделение РНК из тканей производилось с использованием набора для выделения RNeasy mini kit (Qiagen) по протоколу для выделения из клеток животных (animal cells/spin) со следующими модификациями: 1) на шаге 4 добавляли 0.5 объема 96 \% EtOH; 2) после добавления спирта помещали пробирку в термостат на 37 $^\circ$С на 2 минуты, 3) элюция проводилась 30 мкл RNase-free water. Гомогенизацию проводили при помощи растирания пестиком в ступке в жидком азоте. Целостность РНК (RIN) определяли с помощью капиллярного электрофореза на приборе Bioanalyzer 2100 (Agilent). Для дальнейшей работы использовали образцы с RIN не менее 7. Выделение РНК проводили в центре коллективного пользования в области геномики Сколтеха (https://www.skoltech.ru/research/en/shared-resources/gcf-2/).

\section{Амплификация отдельных генов методом полимеразной цепной реакции (ПЦР)}

В первой части работы для лучшего разрешения филогенетического дерева Arvicolinae, необходимого для анализа изменчивости гена \textit{CYTB}, наряду с самим геном были использованы и семь ядерных генов: ген рака груди 1 (\textit{BRCA1}), экзон 11; ген рецептора гормона роста (\textit{GHR}), экзон 10; фрагмент гена лецитин-холестерин-ацилтрансферазы (\textit{LCAT}), экзоны 2-5 и интроны 2-4; ген белка-супрессора опухолей (\textit{PT53}), экзоны 5-7 и интроны 5-6; ген интерфоторецепторного ретиноид-связывающего белка (\textit{IRBP}); ген фактора фон Виллебранда (\textit{vWF}), экзон 28; и ген кислой фосфатазы типа V (\textit{Acp5}), экзоны 2 и 3. Все использованные праймеры перечислены в таблице \ref{primers}. Условия амплификации использовали без изменений согласно опубликованным в источниках (табл. \ref{primers}). 

\begin{table}[h!]
\caption{Праймеры, использованные в работе. Внутренние праймеры для секвенирования помечены (*).}\label{primers}

\begin{center}

\begin{tabular}{|c|c|c|c|}
	\hline 
\textbf{Ген}	& \textbf{Праймер} & \textbf{Последовательность (5' - 3')} & \textbf{Ссылка} \\ 
	\hline 
\multirow{2}{*}{\textit{CYTB}}	& L14729  & GACATGAAAAATCATCGTTGTTATT & \cite{Lebedev2007} \\ 
	\cline{2-4} 
		& H15985 & TAGAATGTCAGCTTTGGGTGCT & \cite{Ohdachi2001} \\ 
	\hline 

\multirow{2}{*}{\textit{BRCA1}}	& F180\_arv  & CGGAACAGATGGGCTGAAAGTAAAG & \multirow{2}{*}{\cite{Bannikova2013}}\\ 
\cline{2-3} 
& R1240\_arv & GGCATCTGCTGCAGGTTCTGTGT & \\ 
	\hline 

\multirow{2}{*}{\textit{GHR}}	& arv\_F  & GGCGTTCATGACAACTACAAA & \multirow{2}{*}{\cite{Abramson2009}}\\ 
\cline{2-3} 
& arvic\_R & ATAGCCACACGAGGAGAGGAACT & \\ 
\hline 

\multirow{2}{*}{\textit{LCAT}}	& LCAT F  & CACCATCTTCCTGGATCTCAA & \multirow{2}{*}{\cite{Abramson2009}}\\ 
\cline{2-3} 
& LCAT R & AAGAAATACAGCACATGTAGGCA & \\ 
\hline

\multirow{2}{*}{\textit{PT53}}	& p53 2F  & TYCCCTCAATAAGCTRTTCTGCCA & \multirow{2}{*}{\cite{Petrova2016}}\\ 
\cline{2-3} 
& p53 3R & GTTTATGCCCCCCATGCAGA & \\ 
\hline

\multirow{4}{*}{\textit{IRBP}}	& A3  & CTGATGGGAATGCAAGCAGC & \multirow{4}{*}{\cite{Petrova2016}}\\ 
\cline{2-3} 
& IPL* & GACATCGCCTACATCCTCAAGCA & \\ \cline{2-3} 
& IPR* & CTCAGCTTCTGSAGGTCYAGG & \\ \cline{2-3} 
& B2a & ATGAGGTGYTCYGTGTCCTG & \\
\hline

\multirow{4}{*}{\textit{vWF}}	& V1 & TGTSAACCTYACSTGTGAAGCCTG & \multirow{4}{*}{\cite{Poux2006}}\\ 
\cline{2-3} 
& VIF* & CTACCTCTGTGACCTTGCCCCTGA & \\ \cline{2-3} 
& VIR* & TCAGGGGCAAGGTCACAGAGGTAG & \\ \cline{2-3} 
& W1 & TGCAGGACCAGGTCAGGAGCCTCTC & \\
\hline

\multirow{2}{*}{\textit{Acp5}}	& AP5-120fwd  & AATGCCCCATTCCACACAGC & \multirow{2}{*}{\cite{Steppan2017}}\\ 
\cline{2-3} 
& AP5-564rev & CCCGGGAAATGGCCAATG & \\ 
\hline

\end{tabular} 

\end{center}

\end{table}

\section{Секвенирование}

\subsection{Секвенирование методом Сенгера}

Очистку ПЦР-продуктов проводили с использованием набора Omnix («Омникс», Россия). ПЦР-продукты секвенировали в обоих направлениях с использованием ABI BigDye версии 3.1. на автоматическом капиллярном секвенаторе Genetic Analyzer 3130 (Applied Biosystems) в компании Евроген (https://evrogen.ru/).  

\subsection{Получение коротких прочтений (ридов) методом NGS}

\subsubsection{Получение ридов для сборки митохондриальных геномов}

Для подготовки библиотек ДНК для NGS секвенирования был использован набор NEBNext Ultra II DNA Library Prep Kit for Illumina (New England Biolabs). Приготовление библиотек производилось по протоколу со следующими изменениями: 1)  очистка на магнитных частицах после лигирования проводилась по пункту 3В (without size selection) в соотношении объем образца к объему магнитных частиц -- 1:0,9; 2) число циклов в ПЦР -- 10.
Полученные в результате ПЦР продукты были очищены и концентрированы при помощи магнитных частиц в соотношении объем образца к объему магнитных частиц -- 1:0,9. Элюция проводилась в 20 мкл бидистиллированной воды. Концентрация образцов измерялась на флуориметре Qubit.
Проверка качества полученных библиотек проводилась при помощи Bioanalyzer 2100 Agilent с помощью набора DNA High Sensitivity kit.

Секвенирование проводили на приборе HiSeq4000 (Illumina) со следующими параметрами: длина чтения 75, парные чтения. Качество отсеквенированной ДНК проверяли с помощью Qubit, окончательное распределение длин проверок содержимого адаптера библиотек проводили с помощью Bioanalyzer2100 (Agilent). Демультиплексирование и перевод данных в формат fastq проводили с помощью программы bcl2fastq2. Секвенирование проводили в центре коллективного пользования в области геномики Сколтеха (https://www.skoltech.ru/research/en/shared-resources/gcf-2/).


\subsubsection{Получение ридов для сборки транскриптомов}

Для выделения полиА-РНК из общей фракции РНК и дальнейшей подготовки ДНК-библиотек использовался совмещенный протокол набора NEBNext Poly(A) mRNA Magnetic Isolation Module и набора NEBNext Ultra II Directional RNA Library Prep Kit for Illumina (https://international.neb.com/protocols/). Подготовка проводилось по протоколу со следующими модификациями: 
1) фрагментация образцов проводилась на 94$^\circ$С 10 минут; 
2) синтез первой цепи проводился, используя программу со следующими параметрами: 25$^\circ$С -- 10 минут, 42$^\circ$С -- 30 минут, 70$^\circ$С -- 15 минут; 
3) очистка и концентрирование ДНК проводилось по протоколу при помощи магнитных частиц Ampure XP. Элюция проводилась бидистиллированной водой; 
4) число циклов в ПЦР -- 10 или 15, в зависимости от исходной концентрации РНК. 
Концентрация образцов измерялась на флуориметре Qubit. Проверка качества полученных библиотек проводилась при помощи Bioanalyzer 2100 Agilent с помощью набора DNA High Sensitivity kit. 

Секвенирование проводили на приборе HiSeq4000 (Illumina) со следующими параметрами: длина чтения 75, парные чтения. Качество отсеквенированной ДНК проверяли с помощью Qubit, окончательное распределение длин проверок содержимого адаптера библиотек проводили с помощью Bioanalyzer2100 (Agilent). Демультиплексирование и перевод данных в формат fastq проводили с помощью программы bcl2fastq2. Секвенирование проводили в центре коллективного пользования в области геномики Сколтеха (https://www.skoltech.ru/research/en/shared-resources/gcf-2/).

\section{Сборка и аннотация}

\subsection{Митохондриальные геномы}

Качество сырых чтений оценивалось с помощью FastQC (\cite{Andrews2010}). Очистка от адаптеров или фрагментов с низким качеством проводилась с помощью программы Trimmomatic (\cite{Bolger2014}). 

Поскольку выделение ДНК \textit{Hyperacrius fertilis} производилось из коллекционных образцов прошлого века, это могло сказаться на качестве отсеквенированных последовательностей. Для оценки качества ридов использовали программу mapDamage 2.0 (\cite{Jonsson2013}), отображающую паттерны неправильного включения нуклеотидов и процессов дезаминирования. Это происходит в результате серьезного повреждения ДНК, которое продолжается после смерти организма.

Сборка \textit{de novo} осуществлялась программой plasmid SPAdes (\cite{Bankevich2012}) с настройками по умолчанию. Полученные контиги были отфильтрованы по длине. Для дальнейшей аннотации отобирали контиг с наибольшим сходством по размеру с митохондриальной ДНК для млекопитающих примерно 16 т.п.н. Контиги были аннотированы с помощью веб-сервера MITOS (\cite{Bernt2013}) с настройками по умолчанию и с учетом митохондриального генетического кода позвоночных. Границы генов были проверены и уточнены при сравнении с 21 опубликованной последовательностью митогенома Arvicolinae. Все позиции с низким качеством и покрытием, а также фрагменты, сильно отличающиеся от ранее опубликованных митохондриальных геномов Arvicolinae, были заменены на N вручную. Последовательности белок-кодирующих генов проверяли на содержание преждевременных стоп-кодонов вручную.


\subsection{Транскриптомы}

Качество сырых чтений оценивалось с помощью FastQC (\cite{Andrews2010}). Очистка от адаптеров или фрагментов с низким качеством проводилось с помощью программы Trimmomatic (\cite{Bolger2014}). В работу брали риды только качеством выше 28-29. Tранскриптомы собирали стандартным пакетом Trinity (\cite{Grabherr2011}) с настройками по умолчанию. Поиск кодирующих участков собранных транскриптов проводили в Transdecoder (https://github.com/TransDecoder/). Полученные последовательности очищали от химерных генов программой DIAMOND (\cite{Buchfink2015}), беря для сравнения нуклеотидную базу NCBI (nr). Ортологичные гены определяли с помощью Proteinortho (\cite{Lechner2011}). Полученные ортологичные гены очищали с помощью среды R 3.4.4 (\cite{RCoreTeam2017}), оставляя только универсальные однокопийные ортологи: те гены, которые встречаются в одной копии у всех взятых в анализ видов. 


\section{Выравнивание}

Последовательности генов \textit{CYTB}, ядерных генов и полных митохондриальных геномов были выровнены с помощью программы Mauve 1.1.1 (\cite{Darling2004}), реализованной как плагин Geneious Prime 2019.1 (https://www.geneious.com). Конкатенированная последовательность 13 белок-кодирующих митохондриальных генов была отдельно выровнена с использованием MAFFT 7.222 (\cite{Katoh2014}).

Ядерные ортологичные гены по отдельности были выровнены программой prank (http://wasabiapp.org/software/prank/) с учетом триплетности кодирующей последовательности. Общие для всех фрагменты выравнивания редактировали вручную. Выравнивания конкатенировались в единую последовательность скриптом на языке программирования Python3.

\section{Оценка нуклеотидного состава митохондриальных геномов}

Базовый нуклеотидный состав (процентное содержание каждого из нуклеотидов) был рассчитан в Geneious Prime 2019.1 (https://www.geneious.com). Смещение в нуклеотидном составе (GC-skew) было как $CG_{skew} = \frac{C - G}{C + G}$ (\cite{Arabi2010}; \cite{Hassanin2005}) с помощью пакета BioSeqUtils в BioPython (\cite{Cock2009}) в среде Python 3.0.   

\section{Филогенетическая реконструкция}

\subsection{По гену \textit{CYTB} и семи ядерным генам}

Для анализа изменчивости \textit{CYTB} было построено филогенетическое дерево на основе конкатенированных гена \textit{CYTB} и семи ядерных генов (\textit{BRCA1}, \textit{GHR}, \textit{LCAT},\textit{PT53}, \textit{IRBP},\textit{vWF},\textit{Acp5}). Наилучшее соответствие нескольких моделей замены для каждого гена оценивалось с помощью Treefinder (\cite{Jobb2004}) в соответствии с скорректированным информационным критерием Акаике (AICc). Байесовский анализ на основе конкатенированного выравнивания семи генов с указанием их границ был проведен в MrBayes 3.2.6 (\cite{Ronquist2012}). Каждый анализ начинался со случайных деревьев, и два независимых прогона с 4 Марковскими цепями Монте-Карло (MCMC) выполнялись для 5 миллионов поколений, с выборкой каждого 1000-го поколения. Стандартные отклонения разделенных частот были ниже 0.01, потенциальные коэффициенты уменьшения масштаба были равны 1.0, а сходимость оценивали с помощью статистики ESS в Tracer v1.6 (\cite{Rambaut2014}). Консенсусное дерево было построено на основе деревьев, отобранных после 25\% отжига.

\subsection{По митохондриальным геномам}
Для оценки уровня и направления отбора в митоходриальных генах была создана филогенетическая реконструкция, включавшая 57 видов подсемейства полевочьих и 6 видов в качестве внешней группы. Реконструкция проводилась в программе MrBayes 3.2.2 (\cite{Ronquist2012}), используя 13 белок-кодирующих генов (11 417 п.н.). Внешней группой были выбраны следующие виды: \textit{Akodon montensis} Thomas, 1913; \textit{Peromyscus megalops} Merriam, 1898; \textit{Cricetulus griseus} Milne-Edwards, 1867; \textit{C. kamensis} Satunin, 1903; \textit{C. longicaudatus} Milne-Edwards, 1867; \textit{C. migratorius} Pallas, 1773.
 
Были заданы следующие параметры анализа: nst=mixed и гамма-распределение скоростей замен между сайтами, использовалось деление на партиции по генам. Каждый анализ начинался со случайных деревьев, и два независимых прогона с 4 Марковскими цепями Монте-Карло (MCMC) выполнялись для 5 миллионов поколений, с выборкой каждого 1000-го поколения. Стандартные отклонения разделенных частот были ниже 0.01, потенциальные коэффициенты уменьшения масштаба были равны 1.0, а сходимость оценивали с помощью статистики ESS в Tracer v1.6 (\cite{Rambaut2014}). Консенсусное дерево было построено на основе деревьев, отобранных после исключения первых 25\%. Дерево визуализировали в программе FigTree v1.4 (http://tree.bio.ed.ac.uk/software/figtree/).

\subsection{По ортологичным ядерным генам}

Конкатенированное выравнивание ортологичных генов было очищено от неинформативных сайтов программой Gblocks (\cite{Castresana2000}). Деревья были построены по полному и очищенному выравниванию с помощью программы RAxML (\cite{Stamatakis2014}) с конкатенированием методом Majority Rule с 500 репликами бутстрепа. 

\section{Оценка аминокислотных замен и их распределения}

\subsection{Цитохром \textit{b}}

Достоверные физико-химические аминокислотные изменения между остатками в \textit{CYTB} были обнаружены с использованием модифицированной модели MM01, реализованной в TreeSAAP v3.2 (\cite{Woolley2003}). Восемь категорий (1 -- 8) использовались для представления величины радикальных замен, из которых категории от 6 до 8 указывают на наиболее радикальные замены (\cite{McClellan2001}). Значимые положительные значения (категории 6-8, p.value <0,001) были приняты как признак значимого изменения функции белка.

Распределение синонимичных и несинонимичных замен было рассчитано между подземными и наземными видами на каждом участке отдельно и объединено по координатам доменов. Значимость частоты замещения оценивалась с помощью точного критерия Фишера и поправки на множественное сравнение методом Холма. Все расчеты проводились в программе R v.3.4.4. Координаты доменов были получены на сайте UniProt по координатам \textit{CYTB} \textit{Mus musculus}: https://www.uniprot.org/uniprot/P00158.

Частоты использования аминокислот для каждой позиции гена \textit{CYTB} определяли с использованием всех последовательностей для выбранных видов, чтобы учесть внутривидовые вариации. Этот набор данных включал все имеющиеся последовательности в базе данных Genbank по всем взятым в анализ видам на август 2020 года. Аминокислотные паттерны были рассчитаны с использованием скрипта на Python 3. Тест Фишера для сравнения частот считали с учетом неравных размеров выборки, поправку на множественное сравнение проводили методом Холма в статистической среде R.

\subsection{Митохондриальные геномы и транскриптомы}

Оценка паралельных аминокислотых замен проводилась программой ProtParCon (github.com/iBiology/ProtParCon) с дальнейшим поиском аминокислот, характерных только для подземных грызунов. Поиск осуществлялся с помощью рукописного скрипта на языке программирования Python 3. Оценку достоверности обнаруженных замен проводили с помощью функции ProtParCon с дальнейшей поправкой Холма на множественное сравнение вручную в статистической среде R.  

\section{Оценка уровня отбора}

Количество несинонимичных (\textit{dN}) и синонимичных (\textit{dS}) замен, а также $\omega$ (их соотношение) были рассчитаны несколькими способами: 
\begin{itemize}
	\item[\textbullet] С использованием codeml, реализованного в ete-toolkit (\cite{Huerta-Cepas2016}). Для каждого подземного вида (или группы видов) был проведен отдельный анализ. Подразделение на анализируемые группы производилось по принципу выбора филогенетически ближайших наземных таксонов для сравнения $\omega$ и более удаленных как внешняя группа. Для каждого подземного вида было проведено несколько анализов: с использованием модели со свободными ветвями (b\_free, где ωfrg и ωbkg свободны), нейтральная модель (b\_neut, где ωfrg фиксированно равно единице) и модели M0, где все ветви изменяются с одинаковой скоростью. Для сравнения различных моделей рассчитаны likelihood-ratio test (LRT). Сравнение моделей b\_free и M0 показывает, отличаются ли по уровню замен выделенные ветви (подземные полевочьи) от остальной части дерева (наземные полевочьи). Значения 999 и 0,001 были расценены как ошибки.
	\item[\textbullet] Программой RELAX. Эта система проверки гипотез анализирует, ослаблен или усилен отбор на выделенных ветвях. Достоверное значение K> 1 указывает на то, что уровень отбора увеличен, в то время как K <1 указывает, что наблюдается его ослабление (\cite{Wertheim2015}).
	\item[\textbullet] Алгоритмом aBSREL, который позволяет определить оптимальное число $\omega$ и выявить отдельные ветви, которые находятся под отбором (\cite{Smith2015}).     
\end{itemize}


\section{Моделирование и визуализация третичной структуры \textit{CYTB}}

Моделирование было основано на гомологии структуры из \textit{Lemmus sibiricus} Kerr, 1792 и \textit{Ellobius lutescens} Thomas, 1897. Общая архитектура митохондриальных комплексов незначительно различается по кристаллическим структурам (\cite{Crowley2008}; \cite{Hunte2000}), что оправдывает моделирование по гомологии. За основу брали модель кристаллической структуры  \textit{Bos taurus} Linnaeus, 1758 с разрешением 2,4 Å (1NTM (\cite{Gao2003})). Программа modeller 9.22 (\cite{Webb2016}) использовалась для создания структур комплекса протоколом автомоделирования с настройками по умолчанию. Замены были проанализированы визуально в PyMOL v.2.0 (Schrödinger, LLC). Трансмембранные участки комплекса оценивали с помощью веб-сервера OPM (\cite{Lomize2012}).

\section{Предсказание сайтов фосфорилирования}

NetPhos 3.1 Server (http://www.cbs.dtu.dk/services/NetPhos/) (\cite{Blom2004}) и GPS 5.0 (http://gps.biocuckoo.cn/online.php) (\cite{Xue2011}) были использованы для предсказания изменения статуса фосфорилирования.

%\vspace{0pt plus0.1fill}

Все рукописные скрипты, использованные для анализа данных секвенирования геномной ДНК и РНК, проведения обсчетов и запуска программ, доступны в репозитории:  https://github.com/ZaTaxon/work\_skripts.