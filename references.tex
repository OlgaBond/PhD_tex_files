\clearpage                                  % В том числе гарантирует, что список литературы в оглавлении будет с правильным номером страницы
\phantomsection
\addcontentsline{toc}{chapter}{\bibname}	% Добавляем список литературы в оглавление
%%\hypersetup{ urlcolor=black }               % Ссылки делаем чёрными
%%\providecommand*{\BibDash}{}                % В стилях ugost2008 отключаем использование тире как разделителя 
%%\renewcommand*{\bibfont}{\small}
%
\urlstyle{rm}                               % ссылки URL обычным шрифтом
\printbibliography
%%	\insertbibliofull                          % Подключаем Bib-базы
\urlstyle{tt}                               % возвращаем установки шрифта ссылок URL
%%\hypersetup{ urlcolor={urlcolor} }          % Восстанавливаем цвет ссылок


%\chapter*{Список использованной литературы}

%\bibliography{othercites.bib}

%\begin{enumerate}
%\item Brodsky L.I. et al. Evolutionary regulation of the blind subterranean mole rat, \textit{Spalax}, revealed by genome-wide gene expression // Proc. Natl. Acad. Sci. 2005. Vol. 102, N 47. P. 17047-17052.
%\item	Fang X. Convergent and divergent adaptations of subterranean rodents // PhD thesis. 2015.
%\item	Nevo E., Reig O. Evolution of subterranean mammals at the organismal and molecular levels // Proc. fifth Int. Theriol. Congr. held Rome, Italy,. 1990.
%\item	Nevo E. Mosaic evolution of subterranean mammals: regression, progression, and global convergence. // Oxford Univ. Press. 1999.
%\item	Lacey E., Patton J., Cameron G.N. Life underground: The biology of subterranean rodents. The University of Chicago Press, Chicago, Illinois, 2000. 449 p.
%\item	Begall S., Burda H., Schleich C.E. Subterranean Rodents: News from Underground // Subterranean Rodents. Berlin, Heidelberg: Springer Berlin Heidelberg, 2007. P. 3-9.
%\item	Robert A. Martin. Biochronology of Latest Miocene Through Pleistocene Arvicolid Rodents from the Central Great Plains of North America // Coloquios Paleontol. 2003. Vol. 1. P. 373-383.
%\item	Fejfar O. et al. Microtoid cricetids and the Early history of arvicolids (Mammalia, Rodentia) // Palaeontol. Electron. 2011. Vol. 14, N 3. P. 12.
%\item	Balakirev A.E., Abramov A.V., Rozhnov V.V. Taxonomic revision of Niviventer (Rodentia, Muridae) from Vietnam: a morphological and molecular approach // Russ. J. Theriol. 2012. Vol. 10, N 1. P. 1-26.
%\item	Gomes Rodrigues H., Marivaux L., Vianey-Liaud M. The Cricetidae (Rodentia, Mammalia) from the Ulantatal area (Inner Mongolia, China): New data concerning the evolution of Asian cricetids during the Oligocene // J. Asian Earth Sci. 2012. Vol. 56. P. 160-179.
%\item	Abramson N.I. et al. Supraspecies relationships in the subfamily Arvicolinae (Rodentia, Cricetidae): An unexpected result of nuclear gene analysis // Mol. Biol. 2009. Vol. 43, N 5. P. 834-846.
%\item	Lychev G.F., Savinov P.F. Late Pliocene lagomorphs and rodents of Kiikbay // Mater. Hist. fauna flora Kazakhstan, Almaty. 1974. Vol. 6. P. 39-57.
%\item	Lebedev V. et al. Cryptic variation in mole voles \textit{Ellobius} (Arvicolinae, Rodentia) of Mongolia // Zool. Scr. 2020. Vol. 49, N 5. P. 535-548.
%\item	Martínková N., Moravec J. Multilocus phylogeny of arvicoline voles (Arvicolini, Rodentia) shows small tree terrace size // Folia Zool. 2012. Vol. 61, N 3-4. P. 254-267.
%\item	Martin R.A. et al. Rodent community change at the Pliocene-Pleistocene transition in southwestern Kansas and identification of the \textit{Microtus} immigration event on the Central Great Plains // Palaeogeogr. Palaeoclimatol. Palaeoecol. 2008. Vol. 267, N 3-4. P. 196-207.
%\item	McNab B.K. The Metabolism of Fossorial Rodents: A Study of Convergence // Ecology. 1966. Vol. 47, N 5. P. 712-733.
%\item	White C.R. The Influence of Foraging Mode and Arid Adaptation on the Basal Metabolic Rates of Burrowing Mammals // Physiol. Biochem. Zool. 2003. Vol. 76, N 1. P. 122-134.
%\item	Vleck D. The Energy Cost of Burrowing by the Pocket Gopher \textit{Thomomys bottae} // Physiol. Zool. 1979. Vol. 52, N 2. P. 122-136.
%\item	Gorecki A., Christov L. Metabolic rate of the lesser mole rat // Acta Theriol. (Warsz). 1969.
%\item	Wei D.B. et al. Blood-gas properties of plateau zokor (\textit{Myospalax baileyi}) // Comp. Biochem. Physiol. - A Mol. Integr. Physiol. 2006. Vol. 145, N 3. P. 372-375.
%\item	Avivi A. et al. Adaptive hypoxic tolerance in the subterranean mole rat \textit{Spalax ehrenbergi}: the role of vascular endothelial growth factor // FEBS Lett. 1999. Vol. 452, N 3. P. 133-140.
%\item	Shams I., Nevo E., Avivi A. Ontogenetic expression of erythropoietin and hypoxia‐inducible factor‐1 alpha genes in subterranean blind mole rats // FASEB J. 2005. Vol. 19, N 2. P. 1-14.
%\item	Shams I., Avivi A., Nevo E. Oxygen and carbon dioxide fluctuations in burrows of subterranean blind mole rats indicate tolerance to hypoxic-hypercapnic stresses // Comp. Biochem. Physiol. Part A Mol. Integr. Physiol. 2005. Vol. 142, N 3. P. 376-382.
%\item	Widmer H.R. et al. Working underground: Respiratory adaptations in the blind mole rat // Proc. Natl. Acad. Sci. 1997. Vol. 94, N 5. P. 2062-2067.
%\item	KLEINSCHMIDT T., NEVO E., BRAUNITZER G. The Primary Structure of the Hemoglobin of the Mole Rat (\textit{Spalax ehrenbergi}, Rodentia, Chromosome Species 60) // Hoppe-Seyler´s Zeitschrift für Physiol. Chemie. 1984. Vol. 365, N 1. P. 531-538.
%\item	Arieli R. The atmospheric environment of the fossorial mole rat (\textit{Spalax ehrenbergi}): Effects of season, soil texture, rain, temperature and activity // Comp. Biochem. Physiol. -- Part A Physiol. 1979. Vol. 63, N 4. P. 569-575.
%\item	Gurnett A.M. et al. The myoglobin of rodents:\textit{Lagostomus maximus} (viscacha) and \textit{Spalax ehrenbergi} (mole rat) // J. Protein Chem. 1984. Vol. 3, N 5-6. P. 445-454.
%\item	McNab B.K. The Influence of Body Size on the Energetics and Distribution of Fossorial and Burrowing Mammals // Ecology. 1979. Vol. 60, N 5. P. 1010-1021.
%\item	McNab B.K. Food Habits, Energetics, and the Population Biology of Mammals // Am. Nat. 1980. Vol. 116, N 1. P. 106-124.
%\item	REICHMAN O.J., SMITH S.C. Burrows and burrowing behavior by mammals // Curr. Mammal. 1990. N 2. P. 197-244.
%\item	Darwin C. The expression of the emotions in man and animals // London. 1872.
%\item	Cooper H.M., Herbin M., Nevo E. Ocular regression conceals adaptive progression of the visual system in a blind subterranean mammal // Nature. 1993. Vol. 361, N 6408. P. 156-159.
%\item	Wright S. Pleiotropy in the Evolution of Structural Reduction and of Dominance // Am. Nat. 1964. Vol. 98, N 899. P. 65-69.
%\item	Wilkens H. Genetic Interpretation of Regressive Evolutionary Processes: Studies on Hybrid Eyes of Two Astyanax Cave Populations (Characidae, Pisces) // Evolution (N. Y). 1971. Vol. 25, N 3. P. 530.
%\item	Burda H., Bruns V., Müller M. Sensory adaptations in subterranean mammals. // Prog Clin Biol Res. 1990. N 335. P. 269-293.
%\item	Borghi C.E., Giannoni S.M., Roig V.G. Eye Reduction in Subterranean Mammals and Eye Protective Behavior in C Tenomys // Mastozoología Neotrop. 2002. Vol. 9, N 2. P. 123-134.
%\item	ARMSTRONG J., QUILLIAM T.A. Nerve Endings in the Mole’s Snout // Nature. 1961. Vol. 191, N 4796. P. 1379-1380.
%\item	Heth G. et al. Vibrational communication in subterranean mole rats (\textit{Spalax ehrenbergi}) // Behav. Ecol. Sociobiol. 1987. Vol. 21, N 1. P. 31-33.
%\item	Kimchi T., Etienne A.S., Terkel J. A subterranean mammal uses the magnetic compass for path integration // Proc. Natl. Acad. Sci. 2004. Vol. 101, N 4. P. 1105-1109.
%\item	Lin G. et al. Adaptive evolution of flaky thumb claw and elongated compulsory arousal duration in the subterranean rodent plateau zokor // Ethol. Ecol. Evol. 2011. Vol. 23, N 1. P. 77-80.
%\item	Nevo E. Adaptive Convergence and Divergence of Subterranean Mammals // Annu. Rev. Ecol. Syst. 1979. Vol. 10, N 1. P. 269-308.
%\item	Hildebrand M. Digging in quadrupeds. Cambridge, United Kingdom.: Belknap Press, Cambridge, United Kingdom., 1985.
%\item	Burda H. et al. How to eat a carrot? Convergence in the feeding behavior of subterranean rodents // Naturwissenschaften. 1999. Vol. 86, N 7. P. 325-327.
%\item	Lehmann W.H. The forelimb architecture of some fossorial rodents // J. Morphol. 1963. Vol. 113, N 1. P. 59-76.
%\item	Yalden D.W. The anatomy of mole locomotion // J. Zool. 2009. Vol. 149, N 1. P. 55-64.
%\item	Puttick G.M., Jarvis J.U.M. The functional anatomy of the neck and forelimbs of the Cape golden mole, \textit{Chrysochloris asiatica} (Lipotyphla: Chrysochloridae) // African Zool. 1977. Vol. 12, N 2.
%\item	Andersen D.C. Belowground Herbivory: The Adaptive Geometry of Geomyid Burrows // Am. Nat. 1982. Vol. 119, N 1. P. 18-28.
%\item	Tucker R. The digging behavior and skin differentiations in \textit{Heterocephalus glaber} // J. Morphol. 1981. Vol. 168, N 1. P. 51-71.
%\item	Tomasco I.H., Lessa E.P. Two mitochondrial genes under episodic positive selection in subterranean octodontoid rodents // Gene. Elsevier B.V., 2014. Vol. 534, N 2. P. 371-378.
%\item	Da Silva C.C. et al. Genes and Ecology: Accelerated Rates of Replacement Substitutions in the Cytochrome b Gene of Subterranean Rodents // Open Evol. J. 2009. N 3. P. 17-30.
%\item	Zhang T. et al. Cytochrome b gene selection of subterranean rodent Gansu zokor \textit{Eospalax cansus} (Rodentia, Spalacidae) // Zool. Anz. Elsevier GmbH., 2013. Vol. 252, N 1. P. 118-122.
%\item	Li D.H. Economic Animal Fauna of Qinghai. // Qinghai People’s Publ. House,. 1989. P. 681-682.
%\item	McClellan D.A. et al. Physicochemical evolution and molecular adaptation of the cetacean and artiodactyl cytochrome b proteins // Mol. Biol. Evol. 2005. Vol. 22, N 3. P. 437-455.
%\item	Adkins R.M., Honeycutt R.L. Evolution of the primate cytochrome c oxidase subunit II gene // J. Mol. Evol. 1994. Vol. 38, N 3. P. 215-231.
%\item	Di Rocco F. et al. Rapid evolution of cytochrome c oxidase subunit II in camelids (Tylopoda, Camelidae) // J. Bioenerg. Biomembr. 2006. Vol. 38, N 5-6. P. 293-297.
%\item	Dirmeier R. et al. Exposure of Yeast Cells to Anoxia Induces Transient Oxidative Stress // J. Biol. Chem. 2002. Vol. 277, N 38. P. 34773-34784.
%\item	Tomasco I.H., Lessa E.P. The evolution of mitochondrial genomes in subterranean caviomorph rodents: Adaptation against a background of purifying selection // Mol. Phylogenet. Evol. 2011. Vol. 61, N 1. P. 64-70.
%\item	Luo Y. et al. Mitochondrial genome analysis of \textit{Ochotona curzoniae} and implication of cytochrome c oxidase in hypoxic adaptation // Mitochondrion. 2008. Vol. 8, N 5-6. P. 352-357.
%\item	THOMAS, STEVEN P. SUTHERS R.A. The Physiology and Energetics of Bat Flight // J. Exp. Biol. 1972. N 57. P. 317-335.
%\item	Maina J.N. What it takes to fly: the structural and functional respiratory refinements in birds and bats // J. Exp. Biol. 2000. Vol. 203, N 20. P. 3045 LP - 3064.
%\item	Shen Y.Y. et al. Adaptive evolution of energy metabolism genes and the origin of flight in bats // Proc. Natl. Acad. Sci. U. S. A. 2010. Vol. 107, N 19. P. 8666-8671.
%\item	Schizas N. V. et al. Differential Survival of Three Mitochondrial Lineages of a Marine Benthic Copepod Exposed to a Pesticide Mixture // Environ. Sci. Technol. 2001. Vol. 35, N 3. P. 535-538.
%\item	Takeda K. et al. Replicative Advantage and Tissue-Specific Segregation of RR Mitochondrial DNA Between C57BL/6 and RR Heteroplasmic Mice // Genetics. 2000. Vol. 155, N 2. P. 777 LP - 783.
%\item	Nigro L. Nuclear background affects frequency dynamics of mitochondrial DNA variants in Drosophila simulans // Heredity (Edinb). 1994. Vol. 72, N 6. P. 582-586.
%\item	Hutter C.M., Rand D.M. Competition between mitochondrial haplotypes in distinct nuclear genetic environments: Drosophila pseudoobscura vs. D. persimilis. // Genetics. 1995. Vol. 140, N 2. P. 537 LP - 548.
%\item	Kilpatrick S.T., Rand D.M. Conditional hitchhiking of mitochondrial DNA: frequency shifts of\textit{ Drosophila melanogaster} mtDNA variants depend on nuclear genetic background. // Genetics. 1995. Vol. 141, N 3. P. 1113 LP - 1124.
%\item	Stordeur E. de. Nonrandom partition of mitochondria in heteroplasmic \textit{Drosophila }// Heredity (Edinb). 1997. Vol. 79, N 6. P. 615-623.
%\item	Rand D.M., Clark A.G., Kann L.M. Sexually Antagonistic Cytonuclear Fitness Interactions in \textit{Drosophila melanogaster} // Genetics. 2001. Vol. 159, N 1. P. 173 LP - 187.
%\item	James A.C., Ballard J.W.O. Mitochondrial Genotype Affects Fitness in \textit{Drosophila simulans} // Genetics. 2003. Vol. 164, N 1. P. 187 LP - 194.
%\item	Miller D.N. et al. Evaluation and optimization of DNA extraction and purification procedures for soil and sediment samples // Appl. Environ. Microbiol. 1999. Vol. 65, N 11. P. 4715-4724.
%\item	Lebedev V.S. et al. Molecular phylogeny of the genus Alticola (Cricetidae, Rodentia) as inferred from the sequence of the cytochrome b gene // Zool. Scr. 2007. Vol. 36, N 6. P. 547-563.
%\item	Ohdachi S. et al. Intraspecific phylogeny and geographical variation of six species of northeastern Asiatic Sorex shrews based on the mitochondrial cytochrome b sequences // Mol. Ecol. 2001. Vol. 10, N 9. P. 2199-2213.
%\item	Bannikova A.A. et al. Genetic diversity of \textit{Chionomys} genus (Mammalia, Arvicolinae) and comparative phylogeography of snow voles // Russ. J. Genet. 2013. Vol. 49, N 5. P. 561-575.
%\item	Petrova T. V. et al. Cryptic speciation in the narrow-headed vole\textit{ Lasiopodomys (Stenocranius) gregalis} (Rodentia: Cricetidae) // Zool. Scr. 2016. Vol. 45, N 6. P. 618-629.
%\item	Poux C. et al. Arrival and Diversification of Caviomorph Rodents and Platyrrhine Primates in South America // Syst. Biol. / ed. Soltis P. 2006. Vol. 55, N 2. P. 228-244.
%\item	DeBry R.W., Seshadri S. Nuclear intron sequences for phylogenetics of closely related mammals: an example using the phylogeny of mus // J. Mammal. 2001. Vol. 82, N 2. P. 280-288.
%\item	Andrews S. FastQC: A Quality Control Tool for High Throughput Sequence Data [Online]. // Available online at: http://www.bioinformatics.babraham.ac.uk/projects/fastqc/. 2010.
%\item	Bolger A.M., Lohse M., Usadel B. Trimmomatic: a flexible trimmer for Illumina sequence data // Bioinformatics. 2014. Vol. 30, N 15. P. 2114-2120.
%\item	Bankevich A. et al. SPAdes: A New Genome Assembly Algorithm and Its Applications to Single-Cell Sequencing // J. Comput. Biol. 2012. Vol. 19, N 5. P. 455-477.
%\item	Bernt M. et al. MITOS: Improved de novo metazoan mitochondrial genome annotation // Mol. Phylogenet. Evol. 2013. Vol. 69, N 2. P. 313-319.
%\item	Darling A.C.E. Mauve: Multiple Alignment of Conserved Genomic Sequence With Rearrangements // Genome Res. 2004. Vol. 14, N 7. P. 1394-1403.
%\item	Katoh K., Standley D.M. MAFFT: Iterative Refinement and Additional Methods. 2014. P. 131-146.
%\item	Arabi J. et al. Studying sources of incongruence in arthropod molecular phylogenies: Sea spiders (Pycnogonida) as a case study // C. R. Biol. 2010. Vol. 333, N 5. P. 438-453.
%\item	Hassanin A., Léger N., Deutsch J. Evidence for Multiple Reversals of Asymmetric Mutational Constraints during the Evolution of the Mitochondrial Genome of Metazoa, and Consequences for Phylogenetic Inferences // Syst. Biol. / ed. Collins T. 2005. Vol. 54, N 2. P. 277-298.
%\item	Cock P.J.A. et al. Biopython: freely available Python tools for computational molecular biology and bioinformatics // Bioinformatics. 2009. Vol. 25, N 11. P. 1422-1423.
%\item	Jobb G., Von Haeseler A., Strimmer K. TREEFINDER: A powerful graphical analysis environment for molecular phylogenetics // BMC Evol. Biol. 2004. Vol. 4. P. 1-9.
%\item	Ronquist F. et al. MrBayes 3.2: Efficient Bayesian Phylogenetic Inference and Model Choice Across a Large Model Space // Syst. Biol. 2012. Vol. 61, N 3. P. 539-542.
%\item	Rambaut A., Drummond A.J., Suchard M. Tracer v1. 6. Available from http://beast.bio.ed.ac.uk/Tracer. 2014.
%\item	Woolley S. et al. TreeSAAP: Selection on Amino Acid Properties using phylogenetic trees // Bioinformatics. 2003. Vol. 19, N 5. P. 671-672.
%\item	McClellan D.A., McCracken K.G. Estimating the Influence of Selection on the Variable Amino Acid Sites of the Cytochrome b Protein Functional Domains // Mol. Biol. Evol. 2001. Vol. 18, N 6. P. 917-925.
%\item	R Core Team. R: A language and environment for statistical computing. 2017.
%\item	Yang Z. PAML 4: Phylogenetic Analysis by Maximum Likelihood // Mol. Biol. Evol. 2007. Vol. 24, N 8. P. 1586-1591.
%\item	Nielsen R., Yang Z. Likelihood models for detecting positively selected amino acid sites and applications to the HIV-1 envelope gene // Genetics. 1998. Vol. 148, N 3. P. 929-936.
%\item	Crowley P.J. et al. The role of molecular modeling in the design of analogues of the fungicidal natural products crocacins A and D // Bioorg. Med. Chem. 2008. Vol. 16, N 24. P. 10345-10355.
%\item	Hunte C. et al. Structure at 2.3 Å resolution of the cytochrome \textit{bc1} complex from the yeast \textit{Saccharomyces cerevisiae} co-crystallized with an antibody Fv fragment // Structure. 2000. Vol. 8, N 6. P. 669-684.
%\item	Gao X. et al. Structural Basis for the Quinone Reduction in the bc 1 Complex: A Comparative Analysis of Crystal Structures of Mitochondrial Cytochrome bc 1 with Bound Substrate and Inhibitors at the Q i Site † , ‡ // Biochemistry. 2003. Vol. 42, N 30. P. 9067-9080.
%\item	Webb B., Sali A. Comparative Protein Structure Modeling Using MODELLER // Curr. Protoc. Bioinforma. 2016. Vol. 54, N 1.
%\item	Lomize M.A. et al. OPM database and PPM web server: resources for positioning of proteins in membranes // Nucleic Acids Res. 2012. Vol. 40, N D1. P. D370-D376.
%\item	Blom N. et al. Prediction of post-translational glycosylation and phosphorylation of proteins from the amino acid sequence // Proteomics. 2004. Vol. 4, N 6. P. 1633-1649.
%\item	Xue Y. et al. GPS 2.1: enhanced prediction of kinase-specific phosphorylation sites with an algorithm of motif length selection // Protein Eng. Des. Sel. 2011. Vol. 24, N 3. P. 255-260.
%\item	Ballard J.W.O., Whitlock M.C. The incomplete natural history of mitochondria // Mol. Ecol. 2004. Vol. 13, N 4. P. 729-744.
%\item	Martin A.P. Metabolic rate and directional nucleotide substitution in animal mitochondrial DNA. // Mol. Biol. Evol. 1995.
%\item	Xia X., Hafner M.S., Sudman P.D. On Transition Bias in Mitochondrial Genes of Pocket Gophers // J. Mol. Evol. 1996. Vol. 43. P. 32-40.
%\item	Ballard J.W.O. Comparative Genomics of Mitochondrial DNA in Members of the \textit{Drosophila melanogaster} Subgroup // J. Mol. Evol. 2000. Vol. 51, N 1. P. 48-63.
%\item	Andrews T.D., Jermiin L.S., Easteal S. Accelerated Evolution of Cytochrome b in Simian Primates: Adaptive Evolution in Concert with Other Mitochondrial Proteins? // J. Mol. Evol. 1998. Vol. 47, N 3. P. 249-257.
%\item	Shao Y. et al. Genetic adaptations of the plateau zokor in high-elevation burrows // Sci. Rep. Nature Publishing Group, 2015. Vol. 5. P. 1-11.
%\item	Blier P.U., Dufresne F., Burton R.S. Natural selection and the evolution of mtDNA-encoded peptides: Evidence for intergenomic co-adaptation // Trends Genet. 2001. Vol. 17, N 7. P. 400-406.

%\end{enumerate}