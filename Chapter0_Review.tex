\chapter{Обзор литературы} \label{lit_review}


Подземные грызуны представляют собой широко распространенную филогенетически неоднородную группу, представители которой в разной степени адаптированы к этой среде (\cite{Nevo1990}; \cite{Nevo1999}; \cite{Lacey2000}; \cite{Begall2007}). Подземная среда защищает их от хищников и колебаний параметров окружающей среды, экстремальных условий. Защита особенно важна в периоды повышенной уязвимости животного: отдыха, сна или размножения. С другой стороны, подземная ниша является очень специализированной и сложной: подземные грызуны живут в условиях темноты и лишены большинства сенсорных сигналов, доступных над землей, сталкиваются с низким запасом продовольствия, тратят большое количество энергии на рытье (\cite{Begall2007}) и подвергаются высокой патогенной нагрузке (\cite{Nevo1999}).

%Дать определение подзменым грузунам

\section{Морфо-физиологические адаптации грызунов к подземному образу жизни}

Морфо-физиологические адаптации подземных грызунов, в основном, связаны с теплообменом и адаптациями к гипоксии и гиперкапнии внутри норы, а также с необходимостью эти норы рыть и поддерживать в жилом состоянии (\cite{Gambaryan1957}; \cite{Nevo1999}; \cite{McNab1966}). Уровень специализации, по-видимому, тесно связан с количеством времени, которое животное проводит под землей. 

%В то время как почва изолирует обитателей норы от экстремальных условий надземной среды, она также снижает способность животных сбрасывать тепловую нагрузку, возникающую во время нормальной метаболической деятельности или гиперактивности, связанной с копанием. Норы обладают высокой относительной влажностью, так что испарительное охлаждение или облизывание кожи неэффективно, что еще больше усугубляет проблему потери тепла. Подземные грызуны демонстрируют значительную конвергенцию в общем плане тела и специфических чертах, связанных с копанием (\cite{Nevo1979}). Как отмечает Хильдебранд (\cite{Hildebrand1985}), морфологические модификации связаны, прежде всего, с необходимостью разрыхления и транспортировки строительного материала (почвы). В частности, для этого требуется инструмент для копания, способность производить и передавать значительную силу, способность переносить груз и выносливость. 

%\subsubsection{Уровень базового обмена}


Две конкурирующие, но не исключающие друг друга гипотезы, объясняют уменьшенную скорость базового обмена грызунов (basal metabolic rate, BMR), которые живут и добывают корм под землей (\cite{White2003}). Они предполагают, что сниженный уровень обмена либо компенсирует огромные энергетические затраты поиска подземных кормов (гипотеза стоимости рытья, the cost-of-burrowing hypothesis), либо предотвращает перегрев в системах с закрытыми норами (гипотеза теплового стресса, the thermal-stress hypothesis). Анализ показал (\cite{White2003}; \cite{McNab1966}; \cite{Vleck1979}; \cite{Gorecki1969}), что в засушливых районах роющие грызуны имеют значительно более низкий BMR, чем в нормально увлажненных. BMR подземных и роющих грызунов в увлажненных почвах достоверно не различается, равно как и BMR крупных (> 77 г) подземных и роющих грызунов в засушливых районах. Этот вывод поддерживает гипотезу теплового стресса, поскольку несколько групп имеют одинаковый BMR, не смотря на различия в затратах на добычу пищи. 
%Тем не менее, мелкие (< 77 г) подземные грызуны в засушливых районах имеют значительно более низкий BMR, чем роющие грызуны аналогичного размера. Учитывая высокую удельную массу метаболизма мелких животных, ожидается, что они испытывают сильный энергетический и водный стресс в засушливых условиях. В таких условиях значительно уменьшенный BMR мелких подземных видов может компенсировать огромные энергетические потребности в корме (\cite{White2003}; \cite{McNab1966}; \cite{Vleck1979}; \cite{Gorecki1969}). 

%\subsubsection{Кровяное давление и концентрация газа в крови}


По сравнению с наземными видами, диффузионная способность легких подземных грызунов выше (\cite{Widmer1997}), что позволяет продолжать насыщение крови в легких в условиях гипоксически-гиперкапнической обстановки. Исследования на цокорах \textit{Eospalax fontanierii} Milne-Edwards, 1867 (\cite{Wei2006}) показали различие в содержании кислорода и изменение кровяного давления у особей, проживающих в Тибете. У них наблюдались повышенное количество эритроцитов (8,11 $\pm$ 0,59 х 10\textsuperscript{12}/л) и концентрация гемоглобина (147 $\pm$ 9,85 г/л), но гематокрит (45,9 $\pm$ 3,29\%) и средний эритроцитарный объем (56,67 $\pm$ 2,57 фл) были ниже, чем у обитающей там же \textit{Ochotona curzoniae} Hodgson, 1857, а также лабораторных мышей и крыс. Цокора обладали высокой устойчивостью к изменениям рН в тканях и способностью получать кислород из гипоксически-гиперкапнической среды. В лабораторных условиях \textit{Spalax sp.} Guldenstaedt, 1770 выживал при условиях 3\% O\textsubscript{2} и до 15\% CO\textsubscript{2} по крайней мере 14 часов без каких-либо побочных эффектов или поведенческих изменений (\cite{Avivi1999}; \cite{Shams2005a}; \cite{Shams2005}). При тех же условиях крыса погибала через 2-4 часа (\cite{Avivi1999}). 

%Аминокислотные последовательности гемоглобина слепышей содержат много замен, которые, вероятно, повышают его сродство к кислороду (\cite{KLEINSCHMIDT1984}). Аналогично, гаптоглобин (\cite{Lacey2000}) и миоглобин (\cite{Gurnett1984}) некоторых подземных грызунов также содержат большое количество характерных аминокислотных замен. Это позволяет большему количеству кислорода транспортироваться в ткани из крови даже при низком капиллярном давлении.

%В тканях скелетных мышц \textit{Spalax sp.} плотность митохондрий, капиллярная плотность и концентрация миоглобина значительно увеличена (\cite{Avivi1999}; \cite{Widmer1997}; \cite{Arieli1979}). Это уменьшает диффузионное расстояние кислорода до митохондрий и позволяет эффективно доставлять его в оксислительно-восстановительные комплексы.

%\subsubsection{Размер и форма тела}

Размер тела подземных грызунов зависит от места обитания, энергетических затрат на рытье и терморегуляцию (\cite{Vleck1979}; \cite{McNab1979}). Размер уменьшается при географическом изменении тепловой нагрузки. Положительная корреляция веса с широтой (правило Бергмана) была обнаружена у \textit{Spalax ehrenbergi} Nehring 1897, карликовые популяции которых встречаются в северной пустыне Негев, Израиль (\cite{McNab1966}). Эти экстремальные приспособления уменьшают не только перегрев, но и ежедневные энергетические траты в условиях оганиченных пищевых ресурсов (\cite{McNab1980}). Самый экстремальный адаптивный ответ на тепловую нагрузку заметен у \textit{Heterocephalus glaber} Rüppell 1842, который не только маленького размера и лишен ярко выраженной шерсти, но также имеет самую низкую температуру тела (около 32 $^\circ$C) и самую плохую способность к терморегуляции среди всех известных млекопитающих (\cite{McNab1966}). Форма тела подземных грызунов варьируется от крысоподобной (\textit{Tachyoryctes sp.} Rüppell, 1835) до дорсо-вентрально сплюснутой и колбасообразной формы с плоской головой и короткой массивной шеей (\textit{Spalax sp.}), а хвост -- от длинного (\textit{Heterocephalus sp.}) до полностью редуцированного (\textit{Spalax sp.}). Вес и длина тела колеблются от 35 г и 80 мм (\textit{Heterocephalus glaber}) до 1 500 г и 300 мм (\textit{Bathyergus suillus} Schreber, 1782) или больше (до 4 кг и 535 мм для \textit{Rhizomys sp.} Gray, 1831). Размер тела  изменяется в зависимости от вида, пола (самцы, в основном, крупнее самок), возраста, местообитания. Форма морды изменяется от удлиненной до широкой, плоской и ороговевшей, что позволяет, например,слепышам \textit{Spalax sp.} и батергидам (Bathyergidae) использовать ее для прессования почвы (\cite{Lacey2000}). 

%\subsubsection{Зрение}

Некоторые органы чувств (чаще всего глаза) у подземных грызунов уменьшены, особенно у полностью подземных видов (\cite{REICHMAN1990}). Глаза демонстрируют все стадии регрессии: от средних (\textit{Spalacopus sp.} Wagler, 1832) и маленьких (Bathyergidae; \textit{Tachyoryctes sp.}; \textit{Ctenomys sp.} Blainville, 1826; \textit{Myospalax sp.} Laxmann, 1769; \textit{Ellobius sp.} Fischer, 1814) до полностью покрытых кожей (\textit{Spalax sp.}). Структурно глаза варьируют от почти нормальных до частично (\textit{Tachyoryctes sp.} и Bathyergidae), или полностью редуцированных (\textit{Spalax sp.}) (\cite{Nevo1990}). Острота зрения уменьшается соответственно. Борги с коллегами (\cite{Borghi2002}) обнаружили у разных видов подземных грызунов взаимосвязь между уменьшением глаз и использованием головы в качестве клина при копании. Они также отметили ее между типом рациона (в основном подземные источники пищи, подземные и наземные источники пищи или в основном наземные источники пищи) и размером глаз: глаза меньше, если в рационе больше подземных источники пищи -- из-за различной вероятности нападения хищников. Таким образом, хищничество способствует формированию зрительной системы тех подземных грызунов, которые должны выходить на поверхность в поисках пищи или выталкивать разрыхленную почву на поверхность головой.

%Уменьшенный размер глаз может быть следствием: (а) наличия положительного отбора на глаз меньших размеров, (б) отсутствия факторов, стимулирующих развитие зрительной системы, или (в) отрицательной эволюционной стоимости формирования больших глаз по сравнению с преимуществами поддержания остроты зрения. В соответствии с первой гипотезой, некоторые авторы рассматривали группы с сильно редуцированными глазами как наиболее приспособленных к подземному образу жизни (\cite{Darwin1872}; \cite{Cooper1993}). Вторая гипотеза предполагает, что сокращение глаз будет объясняться отсутствием факторов, благоприятствующих отбору в направлении зрительной системы (\cite{Wright1964}; \cite{Wilkens1971}; \cite{Burda1990}). Наконец, последняя связывает морфологическую регрессию глаз с эволюционным давлением, создаваемым метаболическим бременем поддержания больших глаз и нефункциональной визуальной подсистемы формирования изображений (\cite{Cooper1993}). 



%\subsubsection{Обоняние и сенсорная чувствительность}

Обоняние подземных видов хорошо развито (\cite{REICHMAN1990}), как и тактильное чувство. Последнее позволяет эффективно ориентироваться в норах, особенно при быстром движении назад. Сенсорные элементы включают особые волоски (вибриссы) тела, хвоста и передних конечностей. У \textit{Heterocephalus sp.} есть головные, концевые и половые вибриссы в добавок к сенсорному хвосту. Армстронг и Киллиам (\cite{ARMSTRONG1961}) подробно описали тысячи сенсорных папилл в носах \textit{Myospalax sp.}, которые иннервируются 15-20 нервами каждый. 

Полевые и лабораторные эксперименты подтверждают, что, несмотря на сенсорные ограничения подземных грызунов и дефицит ориентиров в среде обитания, они являются чрезвычайно эффективными навигаторами и избегают ненужных высокозатратных процессов рытья. Перемещение в темноте с использованием независимой от ориентира навигации, например, интеграции пути (называемая также счислением координат, dead reckoning), может играть важную роль для подземных грызунов. \textit{Spalax ehrenbergi} использует этот тип навигации в определении кратчайшего пути сложного лабиринта в лабораторных экспериментах (\cite{Kimchi2004}). Для подземных грызунов визуальные сигналы не имеют такого сильного значения, как для наземных. При этом \textit{Cryptomys sp.} Gray, 1864 и \textit{Spalax sp.} могут использовать направленные сигналы от геомагнитного поля под землей как независимые от света ориентиры направления (\cite{Kimchi2004}). Помимо этого, недавние исследования показали, что \textit{Spalax sp.} чувствителен к сейсмическим колебаниям (\cite{Heth1987}).

%\subsubsection{Навигация}

%\subsubsection{Когти, передние конечности и строение скелета}

Подземные грызуны специализированы в области копания, которое затрагивает зубы или когти. Гамбарян (\cite{Gambaryan1960}) приводит несколько способов рытья: (а) когте-головной: разрыхление земли когтями, выкидывание головой (цокор, прометеева полевка); (б) резцово-головной: разрыхление резцами, выкидывание земли головой (\textit{Spalax}) и (в) резцово-грудной, где выкидывние происходит грудью (\textit{Ellobius}). У видов, которые специализируются на рытье зубами, обычно крупные, сильные, быстро растущие резцы, которые заходят основанием глубоко в челюстную кость до уровня сочленовных отростков. Нижние резцы чаще всего используются для удаления почвы (\cite{Gambaryan1960}; \cite{Hildebrand1985}), и поэтому имеют тенденцию изнашиваться быстрее, чем верхние резцы или моляры. Ряд подземных видов может закрывать свои губы за зубами, не давая почве проникнуть в рот (\cite{Gambaryan1960}; \cite{Nevo1979}; \cite{Hildebrand1985}). У подземных грызунов наблюдается увеличение мышечной массы и объема в частях тела, используемых для различных типов копания. Кроме того, кости, связанные с копанием, становятся широкими и ограненными, отражая их функцию в качестве платформ для прикрепления мощных мышц (\cite{Gambaryan1960}; \cite{Lehmann1963}; \cite{Yalden2009}). Закрепление частей тела, непосредственно не связанных с процессом копания, обеспечивает устойчивую платформу для передачи требуемой силы. Например, плечо должно быть стабилизировано, чтобы служить основой для движения передних конечностей. Большинство подземных видов обладают либо специфическими приспособлениями для такой стабилизации, либо усилением признаков, обнаруженных у их менее специализированных родственников (\cite{Gambaryan1960}; \cite{Puttick1977}).

%(а) крепкими короткими передними конечностями с большой мускулатурой и большими когтями, которые максимизируют создаваемую силу (\textit{Myospalax sp.}), (б) выдающимися быстрорастущими изогнутыми резцами (\textit{Ellobius sp.}, \textit{Spalax sp.}, Bathyergidae, исключая \textit{Bathyergus sp.} Illiger, 1811, который копает передними лапами) или (в) комбинацию обоих (Ctenomyidae). Животное очищает вырытую почву позади него, используя обе пары конечностей: он выбрасывает почву наружу с помощью передних лап, задних лап (Bathyergidae), разворачивается и толкает головой и передней частью стопы или используя как бульдозер голову (\textit{Spalax sp.}). Уникальная процедура рытья, при которой члены колонии работают вместе, есть у \textit{Heterocephalus sp.} (\cite{Nevo1979}).


%Тем не менее, сильно удлиненные когти, полезные при рытье, могут мешать удерживать пищу (в основном в форме палочек и корней растительности). Для этого у некоторых видов есть специально измененный чешуйчатый коготь большого пальца. Более того, он также помогает цокорам в очистке поверхности корней от почвы. 

%\subsubsection{Покровы}

Движение в норах и транспортировка почвы приводят к морфологическим изменениям покровов у подземных грызунов. Полностью подземные виды обычно строят туннели, которые лишь немногим больше, чем их собственный диаметр тела (\cite{Andersen1982}) и, следовательно, поверхность тела находится в постоянном контакте с окружающей средой. Когда особи поворачиваются в норе, они толкают голову между задними ногами или под предплечьем, поворачивая тело на половину оборота. Это движение усиливается за счет дряблой кожи и волос (\cite{Nevo1979}; \cite{Tucker1981}). 

\section{Молекулярные адаптации грызунов к подземному образу жизни}

К началу 21 века стало очевидным, что дальнейшее развитие таких классических дисциплин как зоология и ботаника невозможны без использования молекулярных методов и развития ДНК-технологий и, в том числе, секвенирования. Секвенирование по Сенгеру, также известное как секвенирование первого поколения, было первой основной технологией секвенирования, разработанной Эдвардом Сенгером в 1975 году. С тех пор секвенирование по Сенгеру считалось золотым стандартом секвенирования ДНК в течение двух с половиной десятилетий. Именно с его использованием был выполнен проект «геном человека». Однако, насущная потребность в снижении стоимости секвенирования привела к развитию секвенирования следующего поколения (next-generation sequencing, NGS) или технологий секвенирования с высокой пропускной способностью, которые создают миллионы прочтений за один запуск. Технологии NGS, включая Illumina Solexa, ABI SOLiD, Roche 454 и Helicos, предназначены для значительного снижения стоимости секвенирования ДНК по сравнению со стандартным секвенированием по Сенгеру. На сегодняшний день технологии NGS широко используются в различных контекстах, включая секвенирование экзома, целевое секвенирование, полногеномное секвенирование, профилирование транскриптома и т.д. (\cite{Fang2015}).
 
Молекулярные исследования, проводимые в зоологии на видовом и внутривидовом уровне чаще всего касались отдельных, предположительно селективно нейтральных, участков митохондриального генома: например, генов цитохрома \textit{b} (\textit{CYTB}), цитохром-с-оксидазы 1 (\textit{СОX1}) или фрагментов ядерного генома, полученных методом секвенирования по Сенгеру. В них же искали потенциальные следы адаптаций из-за большого количества сравнительного материала. Технологии же секвенирования NGS серьезным образом изменили характер исследований в области экологии, эволюции и генетики (\cite{Hudson2008}; \cite{Stapley2010}; \cite{Rokas2009}).

Помимо изменений в количестве обрабатываемых данных, развивались сами программы изучения эволюционных процессов, в том числе и оценки уровня, направления и изменения есественного отбора. После введения моделей замены кодонов (\cite{Goldman1994}; \cite{Muse1994}) почти два десятилетия назад возник устойчивый интерес к использованию их для изучения уровня и направления естественного отбора на белок-кодирующие гены. Он обычно оценивается как соотношение ($\omega$) несинонимичных (\textit{dN}) к синонимичным (\textit{dS}) заменам, т.е. нуклеотидных замен, которые приводят к изменению аминокислоты в белковой последовательности по отношению к заменам, которые оставляют аминокислоту исходной. В случае, если это значение превышает единицу, говорят о положительном отборе. В остальных случаях принято говорить об ослаблении (значение от 0 до 1) или консервации последовательности гена отрицательным отбором ($\omega$ < 1) (\cite{Anisimova2009}; \cite{Delport2008}). В конце концов, этот способ анализа  превратился в мощный и популярный подход к поиску признаков естественного отбора в молекулярных данных. 

Янг и Нильсен в 2002 году опубликовали первую модель обсчета отбора на отдельно выделенных таксонах «brach-site», которая включала ограниченные вариации в $\omega$ как между сайтами, так и между ветвями и могла использоваться для обнаружения эпизодического положительного отбора. Косаковски-Понд с коллегами (\cite{KosakovskyPond2011}) показали ограничения этого метода, которые могут привести к появлению ложноположительных результатов и снижению эффективности расчетов. Это происходит, например, в сайтах, которые не соответствуют одной из моделей, в том числе из-за изменения скорости эволюции среди филогенетических линий. Выявленные недостатки привели к усовершенствованию подходов по расчету уровня отбора. Так, Мартин Смит и его коллеги разработали адаптивную модель вероятности случайных эффектов (adaptive branch-site random effects likelihood, aBSREL), ключевым нововведением которой является переменная параметрическая сложность (\cite{Smith2015}). В то же время Вертейм с соавторами представили общую схему проверки гипотез (RELAX) для выявления ослабленного отбора (\cite{Wertheim2015}). 

\subsection{Следы отбора на уровне отдельных генов митохондриального генома}

Изучение адаптаций отдельных генов начиналась с тех локусов, информации о которых было больше всего. Чаще всего ими являлись филогенетические маркеры, что и объясняло наличие отсеквенированных последовательностей у большого количества видов и внутривидовых групп.  Для большинства видов млекопитающих таким локусом был митохондриальный ген цитохрома \textit{b} (\textit{CYTB}).

Цитохром \textit{b} --- ключевой компонент белкового комплекса \textit{bc1}, вовлеченный в окислительное фосфорелирование на мембране митохондрий и синтез АТФ (\cite{Tomasco2014}). Синтез АТФ --- важнейший метаболический процесс. Можно предположить, что вследствие такой важной функциональной нагрузки, этот ген в целом сохраняет консервативную аминокислотную последовательность. Тем не менее, изменения в экологии влекут за собой изменения в метаболической потребности организма в целом и клетки в частности. Это может влиять на направление естественного отбора в генных последовательностях белков, которые участвуют в биохимических путях клеточного дыхания.

Огромную работу по изучению молекулярных адаптаций у подземных грызунов проделала Да Сильва с коллегами (\cite{DaSilva2009}). Они провели анализ четырех семейств с облигатными подземными грызунами: Ctenomyidae, Octodontidae, Bathyergidae и Geomyidae. При оценке уровня отбора у \textit{CYTB} оказалось, что все проанализированные подземные виды показали более высокие значения $\omega$ по сравнению с наземными. Помимо этого, в гене были выявлены отдельные кодоны под положительным отбором, характерные только для отдельных пар (\textit{Ctenomys sp.} и \textit{Spalocopus sp.}; \textit{Ctenomys sp.} и \textit{Heterocephalus sp.}) и две позиции, универсальные для всех подземных грызунов.

В работе Томаско (\cite{Tomasco2014}) у представителей родов \textit{Ctenomys} и \textit{Spalacopus} были исследованы уже два митохондриальных маркера: \textit{CYTB} и вторая субъединица цитохром оксидазы (\textit{COX2}). В них они также обнаружили повышенное значение \textit{dN/dS} в подземных группах по сравнению с наземными сестринскими видами.

Отдельный интерес представляют исследования подземных грызунов в условиях <<двойной гипоксии>>: живущих под землей на высоте около 2 тыс м. Одна из работ посвящена как раз такому грызуну: цокору \textit{Eospalax fontanierii} Milne-Edwards, 1867, эндемичному виду с Лёссового плато Китая (\cite{Zhang2013a}; \cite{Li1989}). Три исследованные области его распространения охватывали градиент высот от уровня моря до 2 тыс м. Сравнение вариативности \textit{CYTB} между группами показало, что у животных, обитающих в высокогорье, аминокислотное разнообразие снижено и ген находится в консервативном состоянии под стабилизирующим отбором, т.е. практически неизменен среди высокогорных особей (следует отметить, что у видов с более низких участков ареала эта изменчивость выражена сильнее). Что интересно, при анализе ядерных генов другого высокогорного подземного вида -- \textit{Myospalax baileyi} Thomas, 1911 -- для особей с более высоких участков Тибета была характерна большая генетическая изменчивость и сильный полиморфизм (\cite{Cai2018}).

Адаптивность гена \textit{CYTB} была показана не только у грызунов, но и у других видов, вынужденных приспосабливаться к особым условиям обитания. Так, исследование китообразных показало у них наличие большого количества замен в матричных и трансмембранных доменах цитохрома, которые могут приводить к изменению его биохимических функций (\cite{McClellan2005}). Тоже было обнаружено и для \textit{Hominidae}. Связывают эти различия с возросшими энергетическими затратами в неокортексe (\cite{Adkins1994}). Верблюды Нового и Старого света (\textit{Tylopoda}, Camelidae), приспособленные к принципиально другим условиям -- экстремальной жаре, также имеют характерные изменения во всех трех субъединицах генах цитохромоксидазы: \textit{COХ1}, \textit{COХ2} и \textit{COХ3} (\cite{DiRocco2006}). 

\subsection{Изменение уровня отбора в митохондриальных белок-кодирующих генах}

Под землей создаются условия с пониженным содержанием кислорода и повышенным содержанием углекислого газа, следовательно, клетки млекопитающих испытывают окислительный стресс из-за гипоксии (\cite{Dirmeier2002}). Исследование на подземных представителях двух близкородственных групп Octodontidae и Ctenomyidae -- \textit{Ctenomys sp.} и \textit{Spalacopus sp.} -- показало наличие положительного отбора во всех митохондриальных генах, за исключением \textit{ND3}. Количество функциональных (приводящих к изменениям в структуре белка) аминокислотных замен на сайт варьировало в зависимости от гена: от 0.03 в \textit{СОХ3} до 0.38 в \textit{ATP8} (\cite{Tomasco2011}). Похожие результаты были получены при исследовании пищухи \textit{Ochotona curzoniae}, вида с Тибетского плато и живущего в условиях гипоксии. В генах митохондриальной ДНК (мтДНК) нашли 186 замен, 15 из которых находятся в крайне консервативной позиции и остаются неизменными у всех других видов, взятых в анализ. Они были найдены в  \textit{CYTB}, \textit{COX1}, \textit{COX2}, \textit{ND2}, \textit{ND3}, \textit{ND5} и \textit{ND6} (\cite{Luo2008}). 

Исследование других млекопитающих с измененными энергетическими затратами также показывает адаптивность митохондриального генома. Летучие мыши, например, используют полет как способ передвижения, что является невероятно затратным с энергетической точки зрения. Как и у птиц, для полета летучих мышей требуется скорость метаболизма, которая в 3-5 раз превышает максимум, наблюдаемый во время тренировок у наземных млекопитающих аналогичного размера (\cite{THOMASSTEVENP.SUTHERS1972}, \cite{Maina2000}).  Оценка отбора на митохондриальных генах летучих мышей показала, что количество несинонимичных замен превышает синонимичные в генах  \textit{ND2}, \textit{ND3}, \textit{ND4L}, \textit{ND4}, \textit{ND5}, \textit{ND6} и \textit{COX3}, а значение $\omega$ (\textit{dN/dS}) достоверно выше (\cite{Shen2010}). 

Помимо перечисленных примеров, прямое влияние митохондриальных гаплотипов на приспособленность к различным условиям отмечено у веслоногих рачков (\cite{Schizas2001}), мышей (\cite{Takeda2000}) и дрозофилы (\cite{Nigro1994}; \cite{Hutter1995}; \cite{Kilpatrick1995}; \cite{Stordeur1997}; \cite{Rand2001}; \cite{James2003}). 


\subsection{Следы отбора в ядерных генах}

Удешевление стоимости методик полногеномного секвенирования позволило искать молекулярные адаптации не только в митохондриальном, но и в ядерном геноме.

Поиск признаков отбора в ядерных генах у двух филогенетически далеких подземных видов -- \textit{Heterocephalus glaber} и \textit{Spalax galili} -- позволил выявить группы генов, которые потенциально могут участвовать в адаптивных процессах к подземному образу жизни: регуляция кровяного давления, гены иммунного ответа и формирования эпителия (\cite{Fang2015}). Другие работы, проведенные как отдельно для вида \textit{Spalax galili}, так и при сравнении \textit{Spalax galili} с \textit{Heterocephalus glaber} и \textit{Fukomys damarensis} Ogilby, 1838 выявили группы генов (\cite{Fang2015}), связанные с устойчивостью клеток в условиях гипоксии, редактированием РНК и ДНК, хромосомными перестройками. Они отличались по уровню отбора от набора видов, взятых для сравнения (крыса, мышь, морская свинка и человек). Так же автором работы было выявлено повышенное количество коротких диспергированных повторов (short interspersed nuclear elements, SINEs) у \textit{Spalax galili}, которые предположительно участвуют в формировании толерантности к гипоксии. 

Огромную работу проделала Калина Дэйвис с коллегами (\cite{Davies2018}), проведя поиск молекулярных адаптаций у сильно удаленных семейств подземных млекопитающих: Rodentia (Bathyergidae и Spalacidae), Afrosoricida (Chrysochloridae), и Eulipotyphla (Talpidae). При анализе более 8 тыс генов были обнаружены от 270 до 480 под положительным отбором для каждой группы. Однако, универсальных генов, которые бы находились под положительным отбором у всех изученных видов, найдено не было. Функции найденных генов, характерных для каждой группы в отдельности, часто были связаны со способностью выживать при низкой концентрации кислорода (например, \textit{PARK7}). Помимо этого, были найдены гены с отличающимся уровнем отбора у подземных млекопитающих, связанные со зрением и слухом, а также иммунной системой. В эти физиологические группы также попали 35 генов с общими параллельными заменами для всех взятых в анализ подземных видов. 


Все обнаруженные авторами статей молекулярные данные и выявленные закономерности согласуются с морфологическими и физиологическими изменениями подземных млекопитающих в целом и грызунов в частности. Из перечисленных исследований можно сделать вывод, что гены митохондриального комплекса могут быть адаптивными, не смотря на консервативные функции белков. Результаты анализов пула ядерных генов очень сильно зависят от количества видов, однако, демонстрируют общие паттерны при изучении подземных грызунов и позволяют выявить гены, которые могут быть потенциально вовлечены в процессы адаптации к подземному образу жизни. Тем не менее, исследования подобного рода на подземных представителях подсемейства полевочьи почти не проводились. 



\section{Подземные грызуны подсемейства Arvicolinae}

Подсемейство полевочьи (Arvicolinae, Rodentia) является самым молодым среди грызунов, быстро эволюционирует и демонстрирует огромное разнообразие форм и групп внутри себя. Полевочьи освоили почти все ландшафты и типы местообитаний в Северном полушарии и демонстрируют самую быструю палеонтологически задокументированную адаптивную радиацию среди современных млекопитающих. Самые ранние полевочьи известны с позднего миоцена (около 7-8 млн лет) как в Евразии, так и в Северной Америке (\cite{RobertA.Martin2003}; \cite{Fejfar2011}). Подсемейство насчитывает около 150 видов, сгруппированных по разным оценкам в 28-30 родов, относящихся к 8-10 трибам (\cite{Musser2005}). Число современных видов в восемь раз больше, чем в сестринском подсемействе хомяков Cricetinae, которое присутствует в летописи еще с раннего миоцена (\cite{GomesRodrigues2012}). Arvicolinae возникли во время серии повторяющихся событий быстрого видообразования, по крайней мере, с тремя <<взрывными>> периодами дивергенции в течение своей эволюционной истории (\cite{Abramson2009}).

Внутри этой группы по меньшей мере пять филогенетически далеких линий, включающих около 10 видов, демонстрируют независимый переход к подземному образу жизни. Прометеева полевка --- \textit{Prometheomys schaposchnikowi} Satunin, 1901 -- представляет собой самую раннюю эволюционную линию среди всех современных полевочьих и является единственной подземной формой среди так называемой первой волны радиации полевок (\cite{Abramson2009}). 
Расхождение между \textit{Prometheomys} и всеми другими видами внутри подсемейства, согласно молекулярному датированию, оценивается примерно в 7 млн лет. Другие подземные линии  относятся к наиболее многочисленной последней волне радиации. 

Слепушонки Ellobiusini –- другие специализированные подземные грызуны подсемества, ранее считались самыми примитивными в группе и рассматривались как самые древние ее представители или даже как полевкозубые хомяки (\cite{Gromov1977}). Неожиданный результат молекулярных исследований подсемейства (\cite{Abramson2009}) показал родство слепушонок с пеструшками (Lagurini), серыми и водяными полевками (Arvicolini), что указывает на принадлежность их к последней радиации подсемейства. Таким образом, переход и специализация к подземному образу жизни произошел у слепушонок довольно поздно, в плиоцене, в отличие от \textit{P. schaposchnikowi}. Подземные слепушонки трибы Ellobiusini представлены единственным родом \textit{Ellobius} Fisher, 1814, который насчитывает пять видов в двух подродах. Ископаемые остатки слепушонок известны на рубеже плиоцена-плейстоцена, примерно 2,5 млн лет (\cite{Lychev1974}). Оценки молекулярного датирования дивергенции подродов слепушонок указывают на период 4,5-4,8 млн лет (\cite{Abramson2009}; \cite{Lebedev2020}; \cite{Abramson2021}). 

Некоторые виды в пределах самой многочисленной трибы Arvicolini (включает по разным оценкам 60-65 видов) также демонстрируют различную степень адаптации к подземному образу жизни, в частности: \textit{Terricola subterraneus} de Selys-Longshamps 1836, \textit{Microtus pinetorum} Le Conte 1830, \textit{Hyperacrius fertilis} True, 1894 и \textit{Lasiopodomys mandarinus} Milne-Edwards 1830. Эти виды принадлежат к разным узлам в пределах трибы (\cite{Abramson2009}; \cite{Martinkova2012}; \cite{Abramson2021}) и не являются потомками одного ближайшего общего предка. Сестринские таксоны каждого вида — обитающие на поверхности или ведущие роющий образ жизни, что указывает на независимый многократный переход к подземному образу жизни. 

Морфологически все перечисленные виды в разной степени адаптированы к подземному образу жизни. Так, ротовой аппарат и передние конечности \textit{Prometheomys} позволяют проследить изменение ниши на подземную, но резцы не изолированы губами и не выдаются вперед. Когти всех передних конечностей, кроме переднего пальца, удлинены. Слепушонки приобретают более явные адапатации подземных грызунов: очень маленькие глаза, отсутствие ушной раковины, сильно выступающие резцы и изолирование ротового отдела губами (\cite{Gromov1977}). \textit{Hyperacrius} обитает в Западных Гималаях на высотах от 1900 до 3600 м над ур.м. Вид считается подземным (или полуподземным), имея густой, короткий и не дифференцированный мех в качестве адаптивного признака (\cite{Gromov1977}). У полевочьих рода \textit{Terricola} переход к подземному образу жизни приводит к короткому густому меху, маленьким глазкам, коротким ушным раковинам, скрытым в мехе, и снижению уровня базового обмена (\cite{Gromov1977}). У других полевочьх, считающихся подземными или полу-подземными, ярко выраженных морфологических адаптаций нет.  

\subsection{Изучение молекулярных адаптаций подземных полевочьих}

Несмотря на очень интенсивную историю исследования подсемейства с применением молекулярных методов, мало что известно относительно молекулярных механизмов их быстрой адаптивной радиации. За относительно недолгий период эволюции этой группы ее представители неоднократно переходили к жизни под землей. Как уже отмечалось выше, предыдущие исследования молекулярных адаптаций к подземному образу жизни выполнялись на немногочисленных и полностью подземных представителях филогенетически далеких таксонов из разных семейств и подотрядов (слепыши, землекопы, туко-туко). Полевки, в свою очередь, предоставляют уникальные возможности для тестирования гипотез об универсальности молекулярных механизмов адаптаций к подземному образу жизни за счет сравнения близкородственных пар подземных и наземных видов в пределах одного семейства и филогенетически удаленных подземных видов. 

Среди подземных полевочьих в плане молекулярных адаптаций активно изучался только вид \textit{Lasiopodomys mandarinus}: авторы (\cite{Sun2018a}) изучили эволюцию генов \textit{CLOCK} и \textit{BAML1}, вовлеченных в циркадные ритмы. Однако, они не обнаружили следов конвергентной эволюции при сравнении с \textit{Heterocephalus glaber}, \textit{Fukomys damarensis} и несколькими видами рода \textit{Spalax}. Затем исследование продолжили, но уже с генами \textit{PER} и \textit{CRY}, также вовлеченными в регуляцию циркадных ритмов (\cite{Sun2018}). В этих генах авторы обнаружили сайты под положительным отбором у представителей \textit{Fukomys damarensis} и видов рода \textit{Spalax}. Цикл изучения циркадных ритмов завершился анализом экспрессии перечисленных генов в глазах и гипоталамусе (\cite{Sun2020}), показав различия между \textit{L. mandarinus} и \textit{Lasiopodomys brandtii} Radde 1861. В 2020 году та же группа авторов провела уже сравнение транскриптомов (\cite{Dong2020}) между этими видами в поиске молекулярных адаптаций к гипоксии, которая возникает при подземном образе жизни. В результате были обнаружены группы генов с измененной экспрессией, связанных с гипоксией и иммунной системой. 

Несмотря на то, что в подсемействе есть другие виды, которые независимо перешли к подземному образу жизни, исследования их молекулярных адаптаций не проводилось. Также не проводились сравнения и анализ конвергенции молекулярных признаков с другими более эволюционно древними подземными грызунами.
 