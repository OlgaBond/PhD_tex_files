
\chapter{Заключение} 

Представители полевочьих, ведущие подземный образ жизни, демонстрируют изменение силы и направления отбора в митохондриальных генах, а также имеют характерные для них параллельные замены в ядерных генах. При анализе только одного митохондриального гена \textit{CYTB} мы показали, что в последовательности повышено соотношение между несинонимичными и синонимичными заменами ($\omega$) у  подземных грызунов по сравнению с наземными и восемь белковых доменов обладают повышенной частотой замен у подземных видов. Результаты, полученные нами при исследовании белок-кодирующих митохондриальных генов, говорят о вовлеченности всего митохондриального генома в адаптивный процесс. Анализы показали повышение уровня отбора в митогеномах у подземных грызунов. Однако, мы наблюдаем различие в количестве генов с достоверным отличием выявленных признаков отбора у разных видов: больше всего таких генов обнаружено у представителей рода \textit{Ellobius}. Обнаруженная у митохондриальных геномов тенденция к ослаблению отбора не подтвердилась на ядерных данных, и нам не удалось найти генов, уровень отбора в которых достоверно бы различался у подземных и наземных грызунов. 

При анализе параллельных аминокислотных замен нами было обнаружено три замены в гене \textit{CYTB}: Ser57Pro, Asp214Asn, и Ile338Val. Замена в сайте 57 также обнаружена у землекопов (семейство Bathyergidae), а в 214 -- у землекопов и туко-туко (род \textit{Ctenomys}). Cреди подземных полевок замена Asp214Asn была обнаружена у \textit{Prometheomys schaposchnikowi}, \textit{Ellobius fuscocapillus} и \textit{Ellobius lutescens}, а такая же у представителей большинства специализированных семейств подземных грызунов (Spalacidae, Bathyergidae). Повторив поиск на ядерных геномах, нам удалось найти два гена с параллельным аминокислотными заменами: \textit{Rad23b} и \textit{Pycr2}. Найденные нами ядерные гены не выявлялись ранее при изучении подземных грызунов. Однако, биохимические пути и процессы, в которые они вовлечены, можно связать с процессами адаптации подземных грызунов. Гены \textit{Rad23b} и \textit{Pycr2} (pyrroline-5-carboxylate reductase 2) связаны с процессами репарации ДНК (\cite{Pohjoismaki2012}) и реакцией на окислительный стресс (\cite{Kuo2015}), соответственно. Гомолог гена \textit{Rad23} был обнаружен при изучении адаптаций к засухе у растений, поскольку его уровень отбора сильно изменялся в сторону положительного (\cite{Zhang2013b}). Изменение его экспрессии также выявлен при анализе устойчивости к холоду \textit{Thinopyrum intermedium} (\cite{Jaikumar2020}). 

Анализ митохондриальных геномов и транскриптомов показал общие характеристики для подземных грызунов: усиление отбора и повышение частоты несинонимичных замен в митохондриальных генах, наличие параллельных аминокислотных замен как в митохондриальных, так и в ядерных генах. Однако, если анализировать каждую подземную линию независимо, видна неоднородность проявления этих признаков. Так, больше всего изменений затронуло род \textit{Ellobius}, а меньше всего -- \textit{Hyperacrius}. Наблюдаемое количество изменений не коррелирует со временем появления таксона. Прометеева полевка (\textit{P.schaposchnikowi}) --- древнейший представитель подсемейства, согласно молекулярным данным ее возраст оценивается в 7 млн лет. Но при этом вид не демонстрирует самое большое количество молекулярных следов адаптации к подземным условиям. Самое большое количество генов с признаками адаптивных изменений как в митохондриальном, так и в ядерном геномах наблюдается у представителей рода \textit{Ellobius}, хотя их переход к подземному образу жизни произошел не ранее плиоцена. Несмотря на это, морфологические изменения представителей этого рода наиболее близки к тем, что наблюдаются у <<модельных>> подземных грызунов семейств Bathyergidae и Spalacidae: выступающие резцы, очень маленькие глаза и изолирование ротового отдела губами (\cite{Gromov1977}). 

В целом, подземные полевочьи повторяют адаптационный путь других подземных грызунов, показывая схожие тенденции. Так, во время многочисленных исследований представителей семейств Bathyergidae и Spalacidae были обнаружены параллельные замены в абсолютно разных генах. Также во многих генах видно ослабление отбора и увеличение количества признаков его проявления по сравнению с наземными. Этот эволюционный тренд подтверждается не только на подземных грызунах, но и на других подземных млекопитающих – Talpidae и Chrysochloridae. Обнаруженные адаптивные изменения в митохондриальном геноме полевочьих (изменения уровня отбора, наличие параллельных замен) совпадают с тенденциями, выявленными ранее у Octodontidae и Ctenomyidae. Таким образом, полученные нами результаты согласуются с гипотезой о том, что переход к подземному образу жизни стимулирует ослабление отбора. При этом количество адаптивных сигналов в геноме и их  выраженность положительно коррелирует  с уровнем специализации вида, а не с временем его возникновения.



\chapter{Выводы} 

\begin{enumerate}
	
\item В гене \textit{CYTB} обнаружены параллельные аминокислотные замены, характерные для подземных грызунов и показано увеличение частоты несинонимичных замен.

\item В большинстве генов митохондриального генома наблюдается процесс ослабления отбора у подземных грызунов. 
 
\item При анализе белок-кодирующих генов были выявлены параллельные замены у подземных грызунов, которые вовлечены в разные биохимические процессы. 

\item Количество генов, в которых произошли изменения уровня отбора или возникли параллельные замены, отличаются для разных видов подземных грызунов и связано скорее с уровнем специализации, а не с эволюционным возрастом вида. 

\item Для подземных представителей эволюционно молодого подсемейства полевочьих характерны те же направления молекулярной адаптивной изменчивости, что и для относительно древних специализированных семейств подземных грызунов.
 

\end{enumerate}


\chapter{Публикации автора по теме диссертации}

Основные положения диссертации изложены в семи опубликованных печатных работах в журналах, индексируемых Web of Science и Scopus.

\begin{itemize} 
	\item[\textbullet] \textbf{Bondareva O. V.}, Abramson N. I. The complete mitochondrial genome of the common pine vole \textit{Terricola subterraneus} (Arvicolinae, Rodentia) //Mitochondrial DNA Part B. – 2019. – Т. 4. – №. 2. – С. 3925-3926;
	\item[\textbullet] Abramson N. I. et al. Phylogenetic relationships and taxonomic position of genus \textit{Hyperacrius} (Rodentia: Arvicolinae) from Kashmir based on evidences from analysis of mitochondrial genome and study of skull morphology //PeerJ. – 2020. – Т. 8. – С. e10364;
	\item[\textbullet]\textbf{ Bondareva O. V.} et al. The complete mitochondrial genomes of three \textit{Ellobius} mole vole species (Rodentia: Arvicolinae) //Mitochondrial DNA Part B. – 2020. – Т. 5. – №. 3. – С. 2485-2487. 
	\item[\textbullet] \textbf{Bondareva O. V.} et al. Searching for signatures of positive selection in cytochrome b gene associated with subterranean lifestyle in fast-evolving arvicolines (Arvicolinae, Cricetidae, Rodentia) //BMC Ecology and Evolution. – 2021. – Т. 21. – №. 1. – С. 1-12.
	\item[\textbullet] \textbf{O. Bondareva}, S. Bodrov, E. Genelt-Yanovskiy, T. Petrova, N. Abramson. Signatures of selection and adaptation to subterranean lifestyle across the transcriptomes of Arvicolinae (Rodentia, Cricetidae)// FEBS Open Bio. - 2021. - 11:P-01.3-17. doi:10.1002/2211-5463.13205
	\item[\textbullet] Abramson N. I. et al. Mitochondrial genome phylogeny of voles and lemmings (Rodentia: Arvicolinae): evolutionary and taxonomic implications //Plos One. – 2021. - 16(11): e0248198.
	\item[\textbullet]  \textbf{Bondareva O.}, Genelt-Yanovskiy, E., Petrova, T., Bodrov, S., Smorkatcheva, A., \& Abramson, N.  Signatures of Adaptation in Mitochondrial Genomes of Palearctic Subterranean Voles (Arvicolinae, Rodentia) //MDPI Genes. – 2021. – V. 12. – №. 12. – P. 1945.
\end{itemize}


%Материалы диссертации были представлены на следующих конференциях, конгрессах и мероприятиях:
%\begin{itemize} 
%	\item[\textbullet] Международные конференции по вычислительной биологии MCCMB (Москва, 2017, 2019);
%	\item[\textbullet] XI Международная конференция по биоинформатике и системной биологии BGRS (Новосибирск, 2018); 
%	\item[\textbullet] 16 Международная конференция Rodens et Spatium (Потсдам, Германия, 2018); 
%	\item[\textbullet] VII Международный конгресс общества генетиков и селекционеров ВОГиС (Санкт-Петербург, 2019);
%	\item[\textbullet] Семинары лабораторий териологии и эволюционной геномики и палеогеномики ЗИН РАН (2017-2021);
%	\item[\textbullet] Отчетная сессия ЗИН РАН (Санкт-Петербург, 2020);
%	\item[\textbullet] Семинары лаборатории эволюционной геномики факультета биоинформатики и биоинженерии МГУ (Москва, 2018, 2021);
%	\item[\textbullet] 45th Federation of the European Biochemical Societies (FEBS) Congress (дистанционно, 2021);
%	\item[\textbullet] XI Съезд Териологического общества при РАН (Москва, 2022);
%	\item[\textbullet] Биоинформатический семинар Университета ИТМО (Санкт-Петербург, 2022).
%\end{itemize} 