
\chapter{Заключение} 

Проведенная нами работа говорит об ослаблении отбора на митохондриальные гены подземных грызунов. Это показано и с помощью оценки уровня отбора разными методами и отдельным подсчетом частоты несинонимичных замен по сайтам и последующим сравнением с частотой замен у наземных грызунов. Также в гене \textit{CYTB} и ядерных генах были обнаружены сайты с параллельными аминокислотными заменами у филогенетически неродственных подземных грызунов.


\chapter{Выводы} 

\begin{enumerate}
	
\item В гене \textit{CYTB} обнаружены паралелльные аминокислотные замены, характерные для подземных грызунов и показано увеличение частоты несинонимичных замен.

\item В большинстве генов митохондриального генома наблюдается процесс ослабления отбора у подземных грызунов. 
 
\item При анализе белок-кодирующих генов были выявлены параллельные замены у подземных грызунов, которые вовлечены в разные биохимические процессы. 

\item Количество генов, в которых произошли изменения уровня отбора или возникли паралельные замены, отличаются для разных видов подземных грызунов и свзяано скорее с уровнем специализации, а не с эволюционным возрастом вида. 

\item Для подземных представителей эволюционно молодого подсемейства полевочьих характерны те же направления молекулярной адаптивной изменчивости, что и для относительно древних специализированных семейств подземных грызунов.
 

\end{enumerate}

