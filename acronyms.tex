\chapter{Список сокращений и условных обозначений}             % Заголовок
\addcontentsline{toc}{chapter}{Список сокращений и условных обозначений}  % Добавляем его в оглавление

\vskip -24pt

\textbf{а.к.} --- аминокислота

\textbf{АТФ} --- аденозинтрифосфат

\textbf{ГГХ} --- гуанидина гидрохлорид. Универсальный антиприонный агент

\textbf{ГДФ} --- гуанозиндифосфат

\textbf{ГТФ} --- гуанозинтрифосфат

\textbf{ДМСО} --- диметилсульфоксид

\textbf{ЭДТА} --- этилендиаминтетрауксусная кислота

\textbf{ЯМР} --- ядерно-магнитный резонанс

\textbf{2TY} --- богатая среда для сверхпродукции рекомбинантных белков в бактериях

\textbf{FOA} --- 5-Fluoroorotic acid. 5-флуорооротовая кислота

\textbf{IPTG} --- изопропил-\textbeta-D-1-тиогалактопиранозид. Аналог аллолактозы, запускающий транскрипцию lac-оперона.

\textbf{LB} --- lysogeny Broth. Богатая среда для роста культур бактерий

\textbf{OR} --- участок олигопептидных повторов в N-домене белка Sup35

\textbf{PNM} --- \PSI no more. Мутации, приводящие к потере приона \PSI в клетке

\textbf{PrP\textsuperscript{C}} --- нормальная форма белка PrP

\textbf{PrP\textsuperscript{Sc}} --- форма PrP, связанная со скрейпи. Белок, претерпевший изменение конформации и прионный переход

\textbf{QN} --- аспарагин-глутамин богатый участок N-домена белка Sup35

\textbf{SC} --- syntetic complete. Полная синтетическая среда.

\textbf{SDS}  --- sodium dodecyl sulfate (лаурилсульфат натрия)

\textbf{WT} --- аллель дикого типа

\textbf{w/v} --- вес к объему

\textbf{YEPD} --- yeast extract peptone dextrose. Полная питательная среда для дрожжей

%\textbf{ДСД} "--- ДНК-связывающий домен 

%\textbf{КФ} "--- кислая фосфатаза

%\textbf{ОТ-ПЦР-РВ} "--- количественная ПЦР в режиме реального времени на матрице реакционной смеси для обратной транскрипции

%\textbf{ПГЛ} "--- Петергофские генетические линии \Scer

%\textbf{п.\,н.} "--- пар нуклеотидов

%\textbf{ПЦМ} "--- проточная цитофлуорометрия

%\textbf{ПЦР} "--- полимеразная цепная реакция

%\textbf{ТФ} "--- транскрипционный фактор

%%GPD-промотор, GAL-промотор??

%\textbf{abslogFC} "--- абсолютное значение двоичного логарифма отношения средних значений экспрессии

%\textbf{DAPI} "--- 4',6-диамино-2-фенилиндол

%\textbf{DEG} "--- гены с различающейся экспрессией

%\textbf{GuHCl} "--- гидрохлорид гуанидина

%\textbf{RiBi} "--- биогенез рибосом (от англ. Ribosome Biogenesis)

%\textbf{RP} "--- рибосомные белки (от англ. Ribosomal Proteins)

%%SDS?
%\textbf{SMC} "--- синтетическая среда на основе раствора солей (от англ. Synthetic Minimal Complete)

%\textbf{SC} "--- синтетическая среда на основе YNB (от англ. Synthetic Complete)

%\textbf{SNV} "--- однонуклеотидное различие (от англ. Single Nucleotide Variations)

%\textbf{ORF} "--- открытая рамка считывания (от. англ Open Reading Frame)

%\textbf{Q/N-богатый} "--- обогащённый остатками глутамина и аспарагина

%\textbf{v/v} --- объём к объёму

%\textbf{YNB} "--- дрожжевая азотная основа (от англ. Yeast Nitrogen Base)

\vskip 24pt

В работе использованы стандартные однобуквенные обозначения нуклеотидов, а также однобуквенные обозначения аминокислот (\cite{IUPAC1984}).

